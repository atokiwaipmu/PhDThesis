\section{Motivation}
Weak gravitational lensing (WL), often referred to as cosmic shear, is a fundamental tool in contemporary cosmology for mapping the distribution of matter in the universe, encompassing both visible and dark matter. As light from distant galaxies travels through the cosmos, it encounters the gravitational fields of massive structures such as galaxies and galaxy clusters. These gravitational fields cause the light to bend slightly, resulting in subtle distortions in the shapes of background galaxies. By measuring and analyzing these distortions, scientists can create detailed maps of the matter distribution, providing evidence for the presence of dark matter, which does not emit light and cannot be detected through traditional astronomical observations.

Understanding the distribution of dark matter is essential, as it accounts for approximately 27\% of the universe's total mass-energy content. Additionally, weak lensing plays a crucial role in investigating dark energy, the enigmatic force driving the accelerated expansion of the universe. Dark energy makes up roughly 70\% of the universe's energy density, and its precise nature remains one of the most pressing questions in cosmology. By studying the patterns of cosmic shear, researchers can constrain key cosmological parameters, such as the matter density parameter ($\Omega_m$) and the amplitude of matter fluctuations ($\sigma_8$), which are vital for developing accurate models of the universe's evolution.

To maximize the potential of weak lensing as a cosmological probe, large-scale surveys are necessary to collect extensive and high-quality data. Several significant surveys have been proposed and are currently in progress, including the Subaru Hyper Suprime-Cam Survey, the Dark Energy Survey (DES), the Vera C. Rubin Observatory Legacy Survey of Space and Time (LSST), and the Nancy Grace Roman Space Telescope. These surveys aim to observe millions of galaxies across vast regions of the sky, generating detailed and precise maps of cosmic shear. The success of these surveys hinges on the ability to accurately measure and interpret weak lensing signals, which demands advanced analysis techniques and robust statistical methods.

A critical component of weak lensing analysis involves the use of mock catalogs, which are simulated datasets designed to mimic real observations. These mock catalogs are typically created by stitching together simulation boxes with periodic boundary conditions to form an observational light cone. This approach enables researchers to produce large volumes of simulated data without incurring prohibitive computational costs. However, this method introduces several artefacts that can compromise the accuracy of the simulated weak lensing signals. Notable artefacts include the finite support effect, the box replication effect, and super-sample covariance (SSC). These artefacts can distort the statistical properties of the convergence field derived from simulations, leading to biased estimates of cosmological parameters and reducing the reliability of weak lensing measurements.

While some studies have investigated the impact of super-sample covariance on higher-order statistics in three-dimensional analyses, the combined effects of SSC with finite support and box replication artefacts remain insufficiently understood. Higher-order statistics, such as the bispectrum, probability density functions (PDFs), and peak and minimum counts, are crucial for capturing the non-Gaussian features of the large-scale structure of the universe. These statistics provide deeper insights into the formation and evolution of cosmic structures beyond what is achievable with two-point statistics alone.

Addressing and mitigating the impact of these simulation artefacts is essential for enhancing the accuracy of cosmological parameter estimations in the next generation of weak lensing surveys, often referred to as Stage IV experiments. Surveys like Euclid, LSST, and the Prime Focus Spectrograph (PFS) require highly precise analyses to meet their scientific goals. Developing reliable analysis pipelines that account for and correct these artefacts will ensure that weak lensing data can be fully utilized to constrain cosmological models and improve our understanding of dark matter and dark energy.

This thesis aims to address these challenges by examining the combined impact of finite support, box replication, and super-sample covariance artefacts on weak lensing statistics. By isolating these artefacts from measurement-related errors and quantifying their contributions to the covariance matrices of higher-order statistics, this research seeks to improve the reliability of cosmological parameter estimations. The findings will provide valuable insights for the design and analysis of future weak lensing surveys, ensuring that the extensive data they collect can be accurately interpreted to advance our knowledge of the universe.

\section{Objectives and Scope}
The primary objective of this thesis is to investigate how specific artefacts in weak lensing simulations affect the accuracy of cosmological parameter estimations. Weak gravitational lensing is a powerful method for mapping the distribution of matter in the universe, including both visible and dark matter, and for probing the properties of dark energy. To fully utilize its potential, it is crucial to understand and minimize the sources of error that can arise during the simulation and analysis processes.

This research focuses on three main artefacts that impact weak lensing simulations: the Finite Support Effect, the Box Replication Effect, and Super-sample Covariance (SSC). Each of these artefacts introduces biases and uncertainties in the covariance matrices used for parameter estimation, potentially leading to incorrect conclusions about cosmological models. By isolating and quantifying the contributions of these artefacts, this thesis aims to enhance the reliability of weak lensing analyses conducted by current and future large-scale surveys.

The scope of this study includes both theoretical and practical aspects of weak lensing. The theoretical component involves a detailed examination of the weak lensing framework, including the mathematical formalism and statistical methods used to analyze cosmic shear signals. On the practical side, the research encompasses the development and execution of N-body simulations to model weak lensing signals and the implementation of techniques to identify and mitigate the effects of the aforementioned artefacts.

Specifically, the objectives of this thesis are:
\begin{itemize}
    \item To provide a comprehensive overview of the basic theory and methodologies of weak gravitational lensing.
    \item To develop and perform N-body simulations that incorporate these artefacts and evaluate their combined effects on higher-order statistics.
    \item To compare simulation results with observational data from existing surveys to validate the findings and propose strategies for mitigating artefact-induced biases.
    \item To provide recommendations for improving simulation techniques and analysis pipelines in future weak lensing surveys to ensure accurate cosmological parameter estimation.
\end{itemize}

By achieving these objectives, this thesis aims to contribute to the refinement of weak lensing analysis methods, thereby supporting the broader goal of understanding the fundamental components and evolution of the universe.

\section{Structure of the Thesis}

This dissertation is composed of twelve chapters, systematically organized to develop the theoretical framework, methodological approaches, and empirical analyses pertinent to the research objectives. The structure is as follows:

\begin{itemize}
    \item \textbf{Chapter 1: Introduction} --- This chapter delineates the research motivation, articulates the specific aims and scope of the study, and provides a comprehensive overview of the thesis structure.

    \item \textbf{Chapter 2: Cosmology} --- A rigorous review of fundamental cosmological principles is presented, commencing with the Einstein Field Equations and the derivation of the Friedmann-Lemaître-Robertson-Walker (FLRW) metric. We delve into linear perturbation theory to analyze small-scale deviations from a homogeneous and isotropic universe. The chapter further examines cosmological distance measures, including luminosity distance and angular diameter distance, essential for interpreting observational data.

    \item \textbf{Chapter 3: Cosmic Microwave Background (CMB) Power Spectrum} --- This chapter provides an in-depth exploration of the CMB power spectrum. The primordial power spectrum is scrutinized, encompassing the dynamics of the inflaton field under the slow-roll approximation and the resultant perturbations leading to spectral indices. We investigate Baryon Acoustic Oscillations (BAOs), emphasizing the dynamics of the photon-baryon plasma and the processes of recombination and decoupling. The chapter concludes with an analysis of Silk damping, elucidating the effects of diffusion damping on the CMB anisotropies.

    \item \textbf{Chapter 4: Weak Gravitational Lensing} --- We expound on the theory of weak gravitational lensing, beginning with the derivation of the lens equation. The lensing matrix is introduced to characterize distortions in the observed images of distant galaxies. The concepts of convergence and its relation to the matter density contrast are elucidated. A detailed discussion on shear, including its decomposition into E-mode and B-mode components, is provided. Additionally, we address reduced shear, galaxy ellipticity, magnification, and magnification bias, which are crucial for interpreting lensing observations.

    \item \textbf{Chapter 5: Power Spectrum Analysis} --- This chapter offers a comprehensive overview of power spectrum analysis as a fundamental tool in cosmology. We discuss the two-point correlation function and its Fourier counterpart, the power spectrum. The matter power spectrum and the convergence power spectrum are examined in detail. Statistical estimators employed for measuring the power spectrum from observational data are introduced and their properties analyzed.

    \item \textbf{Chapter 6: Higher-Order Statistical Measures} --- We investigate higher-order statistical measures that extend beyond the power spectrum to capture non-Gaussian features in the matter distribution. The bispectrum is introduced as a third-order statistic sensitive to phase correlations. We define probability density functions (PDFs), including their normalization, moments, and cumulants, as tools for statistical characterization. The chapter explores peak and minimum counts, detailing methods for their identification and analyzing the properties of peaks in Gaussian random fields. Minkowski Functionals are presented as topological descriptors, along with computational techniques for their evaluation.

    \item \textbf{Chapter 7: Covariance Analysis} --- This chapter examines the covariance properties of various statistical estimators. We analyze the covariance matrices of the matter power spectrum and the angular power spectrum, discussing their dependence on survey parameters and cosmological models. The covariance of higher-order statistics is also explored. We introduce the Fisher matrix formalism for forecasting parameter constraints from statistical measurements, emphasizing its role in cosmological parameter estimation.

    \item \textbf{Chapter 8: Astronomical Surveys} --- An introduction to astronomical surveys is provided, outlining their objectives and the scientific questions they aim to address. We discuss the methodologies of imaging versus spectroscopic surveys, and the advantages and limitations of ground-based versus space-based observations. The classification of surveys into Stage-III and Stage-IV is explained in the context of their scope and technological capabilities. Detailed descriptions of specific surveys, including the Subaru Hyper Suprime-Cam, the Dark Energy Survey, the Vera C. Rubin Observatory (formerly LSST), and the Nancy Grace Roman Space Telescope, are presented.

    \item \textbf{Chapter 9: Cosmological Simulations} --- This chapter focuses on the cosmological simulations utilized in this study. We discuss the generation of initial conditions based on the primordial power spectrum and the implementation of perturbation theory. The fundamentals of $N$-body simulations are covered, including mass assignment schemes and force calculation methods. We explain various computational algorithms such as Particle-Mesh (PM), Tree-Particle-Mesh (Tree-PM), and Particle-Particle Particle-Mesh (P\textsuperscript{3}M) codes, highlighting their computational efficiencies and accuracies.

    \item \textbf{Chapter 10: Methodological Framework} --- We detail the methodological approaches employed in our research. An overview of the simulations used, specifically the \textbf{BigBox} and \textbf{Tiled} simulations, is provided, including their parameters and cosmological models. The process of generating convergence maps from the simulated data and incorporating observational noise is thoroughly explained. Techniques for patch extraction, the discretized pixel approach, and Gaussian smoothing are discussed as preprocessing steps. The chapter also covers the statistical measurements performed, computational methods for their evaluation, covariance matrix estimation techniques, and data decomposition methods utilized in the analysis.

    \item \textbf{Chapter 11: Results and Analysis} --- This chapter presents the results derived from our analyses. We compare mean values obtained from different simulations to assess consistency and systematic effects. The impact of simulation parameters on covariance matrices is evaluated. The effectiveness of various statistical measures, including power spectra and higher-order statistics, in capturing cosmological information is critically assessed. We present the implications of these findings for cosmological parameter estimation.

    \item \textbf{Chapter 12: Discussion and Conclusions} --- Finally, we discuss the broader implications of our findings within the context of contemporary cosmological models. We summarize the key conclusions drawn from the research, emphasizing contributions to the field. Potential avenues for future research are suggested, highlighting how the methodologies and results of this study can be extended or applied to other areas in cosmology.
\end{itemize}

Additionally, the thesis includes an \textbf{Appendix} section containing supplementary material such as detailed mathematical derivations, simulation parameters, and additional figures that support the main text. A comprehensive \textbf{References} section lists all the sources cited throughout the thesis, ensuring proper attribution and facilitating further reading.

This structured approach ensures a logical progression from theoretical foundations to practical applications and original research, providing a thorough exploration of the challenges and solutions related to weak gravitational lensing simulations in cosmology.


