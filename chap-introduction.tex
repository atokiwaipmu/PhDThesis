\section{Motivation}
Weak gravitational lensing (WL), or cosmic shear, is essential for mapping the universe's matter distribution, including dark matter and dark energy, which constitute about 27\% and 70\% of its mass-energy, respectively \citep{2001PhR...340..291B, 2010CQGra..27w3001B, 1992grle.book.....S, 1998AJ....116.1009R,1999ApJ...517..565P, 2013MNRAS.432.2433H}. Subtle distortions in light from distant galaxies, caused by massive structures, allow scientists to infer dark matter distribution and study dark energy, constraining cosmological parameters like the matter density ($\Omega_m$) and fluctuation amplitude ($\sigma_8$).

Large-scale surveys such as the Subaru Hyper Suprime-Cam \citep{2018PASJ...70S...4A}, Dark Energy Survey (DES) \citep{2018ApJS..239...18A}, Vera C. Rubin Observatory LSST \citep{2019ApJ...873..111I}, and Nancy Grace Roman Space Telescope \citep{2015arXiv150303757S} collect extensive WL data, requiring precise measurements and advanced analysis techniques.

Mock catalogs, simulated datasets mimicking real observations, are crucial for WL analysis. While stacking multi-resolution boxes to generate non-repeating lightcones has been successful \citep{2015MNRAS.448.2987F, 2015MNRAS.453.1513C, 2017ApJ...850...24T, 2019ApJ...875...69D}, it struggles with high-redshift resolution. Alternatively, repeating a single box retains high-redshift resolution and is computationally efficient \citep{2010ApJ...709..920S, 2018JCAP...03..049L, 2020JCAP...10..012S, 2024MNRAS.530.5030O} but introduces artefacts like finite support \citep{2015MNRAS.450.2857H}, box replication \citep{2024MNRAS.534.1205C}, and super-sample covariance (SSC) \citep{2018JCAP...10..053B}, potentially distorting WL signals and biasing cosmological estimates. The combined effects of these artefacts on higher-order statistics remain unclear, despite their importance in capturing non-Gaussian large-scale structures \citep{2013PhRvD..88l3002P, 2015PhRvD..91j3511P, 2015PhRvD..91f3507L}.

Mitigating these artefacts is necessary for accurate parameter estimation in next-generation surveys like Euclid \citep{2010arXiv1001.0061R}, LSST, and the Prime Focus Spectrograph (PFS) \citep{2016SPIE.9908E..1MT}. Developing analysis pipelines to correct these issues will enhance WL data's utility in constraining cosmological models.

\section{Structure of the Thesis}

This dissertation is organized into eleven chapters that develop the theoretical framework, methodologies, and empirical analyses pertinent to the research objectives:

\begin{itemize}
    \item \textbf{Chapter 1: Introduction} --- Outlines the research motivation, aims, and thesis structure.
    
    \item \textbf{Chapter 2: Cosmology} --- Reviews fundamental cosmological principles, including the Einstein Field Equations, the FLRW metric, linear perturbation theory, and cosmological distance measures.
    
    \item \textbf{Chapter 3: Weak Gravitational Lensing} --- Explores the theory of weak gravitational lensing, covering the lens equation, lensing matrix, convergence, shear, reduced shear.
    
    \item \textbf{Chapter 4: Power Spectrum Analysis} --- Provides an overview of power spectrum analysis, discussing the two-point correlation function, matter power spectrum, convergence power spectrum, and statistical estimators.
    
    \item \textbf{Chapter 5: Higher-Order Statistical Analysis} --- Investigates higher-order statistics like the bispectrum, probability density functions, peak and minimum counts, and Minkowski Functionals for capturing non-Gaussian features.
    
    \item \textbf{Chapter 6: Covariance Analysis} --- Examines the covariance properties of statistical estimators, including covariance matrices of power spectra and higher-order statistics, and introduces the Fisher matrix formalism.
    
    \item \textbf{Chapter 7: Astronomical Surveys} --- Introduces astronomical surveys, discussing imaging vs. spectroscopic methods, ground-based vs. space-based observations, and details specific surveys like the Subaru Hyper Suprime-Cam, DES, LSST, and the Nancy Grace Roman Space Telescope.
    
    \item \textbf{Chapter 8: Cosmological Simulations} --- Focuses on cosmological simulations, covering initial conditions, $N$-body simulations, mass assignment schemes, and computational algorithms like PM, Tree-PM, and P$^3$M codes.
    
    \item \textbf{Chapter 9: Methodological Framework} --- Details the methodologies used, including simulations (\textbf{BigBox} and \textbf{Tiled}), convergence map generation, incorporation of observational noise, statistical measurements, covariance matrix estimation, and data decomposition methods.
    
    \item \textbf{Chapter 10: Results and Analysis} --- Presents and analyzes the results, assessing the impact of simulation parameters on covariance matrices and evaluating the effectiveness of statistical measures for cosmological parameter estimation.
    
    \item \textbf{Chapter 11: Discussion and Conclusions} --- Discusses the implications of the findings, summarizes key conclusions, and suggests potential avenues for future research.
\end{itemize}