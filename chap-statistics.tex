\section{Convergence Power Spectrum}
The matter power spectrum, \( P(k) \), is a fundamental quantity in cosmology that characterizes the distribution of dark matter density fluctuations in Fourier space. It is defined as the Fourier transform of the two-point correlation function of the dark matter density field, \( \delta(\mathbf{x}) \) \citep{2001PhR...340..291B}. Mathematically, the matter power spectrum is expressed as:
\begin{equation}
    \langle \tilde{\delta}(\mathbf{k}) \tilde{\delta}(\mathbf{k}') \rangle = (2\pi)^3 \delta^{(3)}(\mathbf{k} + \mathbf{k}') P(k),
    \label{eq:matter_power_spectrum}
\end{equation}
where \( \tilde{\delta}(\mathbf{k}) \) represents the Fourier transform of the density contrast \( \delta(\mathbf{x}) \), and \( \delta^{(3)} \) is the three-dimensional Dirac delta function ensuring statistical isotropy and homogeneity.

In the context of weak gravitational lensing, the matter power spectrum \( P(k) \) is not directly observable. Instead, observations yield the angular power spectrum of the convergence field, \( C_{\ell}^{\kappa\kappa} \), which encapsulates the statistical properties of the convergence \( \kappa(\boldsymbol{\theta}) \) across the sky \citep{2001PhR...340..291B}. The convergence power spectrum, \( P_{\kappa}(\ell) \), is defined through the relation:
\begin{equation}
    \langle \tilde{\kappa}(\boldsymbol{\ell}) \tilde{\kappa}(\boldsymbol{\ell}') \rangle = (2\pi)^2 \delta^{(2)}(\boldsymbol{\ell} + \boldsymbol{\ell}') P_{\kappa}(\ell),
    \label{eq:convergence_power_spectrum}
\end{equation}
where \( \tilde{\kappa}(\boldsymbol{\ell}) \) is the Fourier transform of the convergence field \( \kappa(\boldsymbol{\theta}) \), and \( \delta^{(2)} \) is the two-dimensional Dirac delta function.

The convergence field \( \kappa(\boldsymbol{\theta}) \) can be expressed as a weighted projection of the matter density contrast along the line of sight (see Eq.~\eqref{eq:convergence_integral}):
\begin{equation}
    \kappa(\boldsymbol{\theta}) = \int_0^{\chi_s} d\chi \, W(\chi) \, \delta_m\left(\chi \boldsymbol{\theta}, \chi\right),
    \label{eq:kappa_projection}
\end{equation}
where \( W(\chi) \) is the lensing kernel, \( \chi \) is the comoving radial distance.

Recognizing the Fourier transform of the matter density field \( \delta_m\left(\chi \boldsymbol{\theta}, \chi\right) \), we write:
\begin{equation}
    \delta_m\left(\chi \boldsymbol{\theta}, \chi\right) = \int \frac{d^3\mathbf{k}}{(2\pi)^3} \tilde{\delta}(\mathbf{k}) e^{i \mathbf{k} \cdot \mathbf{x}},
    \label{eq:delta_m_fourier}
\end{equation}
where \( \mathbf{x} = (\chi \boldsymbol{\theta}, \chi) \) is the comoving position vector. Substituting this into the Fourier transform of the convergence field \( \tilde{\kappa}(\boldsymbol{\ell}) \), we obtain:
\begin{eqnarray}
    \tilde{\kappa}(\boldsymbol{\ell}) &=& \int_0^{\chi_s} d\chi \, W(\chi) \int d\boldsymbol{\theta} \, e^{-i \boldsymbol{\ell} \cdot \boldsymbol{\theta}} \delta_m\left(\chi \boldsymbol{\theta}, \chi\right) \nonumber \\
    &=& \int_0^{\chi_s} d\chi \, W(\chi) \int \frac{d^3\mathbf{k}}{(2\pi)^3} \tilde{\delta}(\mathbf{k}) e^{i k_\parallel \chi} \int d\boldsymbol{\theta} \, e^{-i \boldsymbol{\ell} \cdot \boldsymbol{\theta}} e^{i \chi \mathbf{k}_\perp \cdot \boldsymbol{\theta}} \nonumber \\
    &=& \int_0^{\chi_s} d\chi \, W(\chi) \int \frac{dk_\parallel}{2\pi} \frac{d^2\mathbf{k}_\perp}{(2\pi)^2} \tilde{\delta}_m(k_\parallel, \mathbf{k}_\perp) e^{i k_\parallel \chi} \int d\boldsymbol{\theta} \, e^{-i (\boldsymbol{\ell} - \chi \mathbf{k}_\perp) \cdot \boldsymbol{\theta}} \nonumber \\
    &=& \int_0^{\chi_s} d\chi \, \frac{W(\chi)}{\chi^2} \int \frac{dk_\parallel}{2\pi} \tilde{\delta}_m\left(k_\parallel, \frac{\boldsymbol{\ell}}{\chi}\right) e^{i k_\parallel \chi},
    \label{eq:kappa_fourier_final}
\end{eqnarray}
where \( \mathbf{k}_\perp \) and \( k_\parallel \) are the components of \( \mathbf{k} \) perpendicular and parallel to the line of sight, respectively.

Next, we compute the ensemble average of the Fourier transform of the convergence field:
\begin{eqnarray}
    \langle \tilde{\kappa}(\boldsymbol{\ell}) \tilde{\kappa}(\boldsymbol{\ell}') \rangle &=& \int_0^{\chi_s} d\chi \, \frac{W(\chi)}{\chi^2} \int_0^{\chi_s} d\chi' \, \frac{W(\chi')}{\chi'^2} \int \frac{dk_\parallel}{2\pi} \int \frac{dk'_\parallel}{2\pi} \langle \tilde{\delta}_m\left(k_\parallel, \frac{\boldsymbol{\ell}}{\chi}\right) \tilde{\delta}_m\left(k'_\parallel, \frac{\boldsymbol{\ell}'}{\chi'}\right) \rangle e^{i k_\parallel \chi} e^{i k'_\parallel \chi'} \nonumber \\
    &=& \int_0^{\chi_s} d\chi \, \frac{W(\chi)}{\chi^2} \int_0^{\chi_s} d\chi' \, \frac{W(\chi')}{\chi'^2} \int \frac{dk_\parallel}{2\pi} e^{i k_\parallel (\chi - \chi')} (2\pi)^2 \delta^{(2)}\left(\frac{\boldsymbol{\ell}}{\chi} + \frac{\boldsymbol{\ell}'}{\chi'}\right) P_m\left(\sqrt{k_\parallel^2 + \frac{\ell^2}{\chi^2}}\right) \nonumber 
    \label{eq:kappa_power_spectrum}
\end{eqnarray}
where in the last step we have applied the Limber approximation \citep{1954ApJ...119..655L}, which assumes \( k_\parallel \ll \ell/\chi \).
Under the Limber approximation, the integral simplifies as:
\begin{equation}
    \int \frac{dk_\parallel}{2\pi} e^{i k_\parallel (\chi - \chi')} P_m\left(\sqrt{k_\parallel^2 + \frac{\ell^2}{\chi^2}}\right) \approx P_m\left(\frac{\ell}{\chi}\right) \delta(\chi - \chi'),
    \label{eq:limber_approximation}
\end{equation}
where \( P_m(k) \) is evaluated at \( k = \ell/\chi \).
Substituting this into Eq.~\eqref{eq:kappa_power_spectrum}, we obtain:
\begin{equation}
    \langle \tilde{\kappa}(\boldsymbol{\ell}) \tilde{\kappa}(\boldsymbol{\ell}') \rangle = (2\pi)^2 \delta^{(2)}(\boldsymbol{\ell} + \boldsymbol{\ell}') \int_0^{\chi_s} d\chi \, \frac{W^2(\chi)}{\chi^2} P_m\left(\frac{\ell}{\chi}; \chi\right),
    \label{eq:kappa_power_spectrum_final}
\end{equation}
where \( P_m\left(\frac{\ell}{\chi}; \chi\right) \) denotes the matter power spectrum evaluated at wavenumber \( k = \ell/\chi \) and at the comoving distance \( \chi \).
Finally, equating this result with the definition of the convergence power spectrum in Eq.~\eqref{eq:convergence_power_spectrum}, we derive the expression for \( C_{\ell}^{\kappa\kappa} \):
\begin{equation}
    C_{\ell}^{\kappa\kappa} = \int_0^{\chi_s} d\chi \, \frac{W^2(\chi)}{\chi^2} P_m\left(\frac{\ell}{\chi}; \chi\right).
    \label{eq:convergence_power_spectrum_final}
\end{equation}
This relation demonstrates how the observable convergence power spectrum \( C_{\ell}^{\kappa\kappa} \) is sourced by the underlying matter power spectrum \( P_m(k; \chi) \) integrated along the line of sight.

\section{Bispectrum}
The bispectrum, \( B(k) \), serves as the Fourier counterpart to the three-point correlation function and is the lowest-order statistical quantity capable of characterizing non-Gaussianity in the matter distribution \citep{2002PhR...367....1B}. While the power spectrum effectively captures Gaussian fluctuations through two-point statistics, the bispectrum provides deeper insights by incorporating three-point correlations, thereby unveiling more complex structures in the cosmic density field \citep{1999ApJ...517..531S, 2004MNRAS.348..897T}.

Analogous to the angular power spectrum, the convergence bispectrum can be expressed as the ensemble average of three Fourier-transformed convergence modes, \( \tilde{\kappa} \) \citep{2005PhRvD..72h3001D}:
\begin{equation}
    \langle \tilde{\kappa}(\mathbf{\ell}_1) \tilde{\kappa}(\mathbf{\ell}_2) \tilde{\kappa}(\mathbf{\ell}_3) \rangle = (2\pi)^2 \delta_{D}(\mathbf{\ell}_1 + \mathbf{\ell}_2 + \mathbf{\ell}_3) B^\kappa_{\ell_1 \ell_2 \ell_3},
    \label{eq:convergence_bispectrum_def}
\end{equation}
Building upon the derivations analogous to Equations~\eqref{eq:kappa_fourier_final} through \eqref{eq:kappa_power_spectrum_final}, the convergence bispectrum can be expressed as:
\begin{equation}
    B_{\ell_1 \ell_2 \ell_3}^\kappa = \int_0^{\chi_s} d\chi \, \frac{W^3(\chi)}{\chi^4} B_m\left( \frac{\ell_1}{\chi}, \frac{\ell_2}{\chi}, \frac{\ell_3}{\chi}; \chi \right),
    \label{eq:convergence_bispectrum}
\end{equation}
where \( B_m(k_1, k_2, k_3, z) \) denotes the matter bispectrum at redshift \( z \), and \( W(\chi) \) is the lensing kernel. 

The bispectrum depends not only on the magnitudes of the wavevectors but also on the shapes formed by the triplet \( (\boldsymbol{k}_1, \boldsymbol{k}_2, \boldsymbol{k}_3) \), constrained by the condition \( \boldsymbol{k}_1 + \boldsymbol{k}_2 + \boldsymbol{k}_3 = 0 \). Different triangle configurations (e.g. equilateral, squeezed, isoceles) probe different physical processes and scales in the Universe \citep{2005PhRvD..72h3001D}.

\section{Probability Density Functions} \label{sec:pdfs}
The Probability Density Function (PDF) of the convergence field, $\kappa$, provides a fundamental statistical characterization of the field's one-point distribution. By encompassing all moments and cumulants, the PDF captures both Gaussian and non-Gaussian features intrinsic to the convergence field.

Formally, the PDF \( P(\kappa) \) is defined such that:
\begin{equation}
    P(\kappa) \, d\kappa = \mathrm{Prob}(\kappa \leq \kappa' \leq \kappa + d\kappa),
\end{equation}
where \(\mathrm{Prob}\) denotes the probability that the convergence \(\kappa'\) lies within the interval \([\kappa, \kappa + d\kappa]\).

For a discrete set of convergence measurements \(\{\kappa_i\}_{i=1}^{N_{\mathrm{pix}}}\) obtained from \(N_{\mathrm{pix}}\) pixels, the PDF can be represented using the Dirac delta function \(\delta_D\):
\begin{equation}
    P(\kappa) = \frac{1}{N_{\mathrm{pix}}} \sum_{i=1}^{N_{\mathrm{pix}}} \delta_D(\kappa - \kappa_i).
    \label{eq:pdf_delta}
\end{equation}
This expression effectively constructs the PDF by summing over all pixel values, assigning a weight to each convergence measurement \(\kappa_i\) at its exact value.

In practical applications, however, the Dirac delta function is not computationally feasible. Instead, we approximate the PDF by discretizing the convergence values into bins of finite width \(\Delta\kappa\). This leads to a binned estimator:
\begin{equation}
    P(\kappa) \approx \frac{1}{N_{\mathrm{pix}} \Delta\kappa} \sum_{i=1}^{N_{\mathrm{pix}}} \Theta\left(\left|\kappa_i - \kappa\right| \leq \frac{\Delta\kappa}{2}\right),
    \label{eq:pdf_binned}
\end{equation}
where \(\Theta(x)\) is the Heaviside step function. This estimator counts the number of convergence \(\kappa_i\) that fall within each bin centered at \(\kappa\), normalizing by the total number of pixels and the bin width \(\Delta\kappa\).

To facilitate comparison across different datasets, it is common to normalize the convergence values by their standard deviation $\sigma_\kappa$. The standardized convergence $\tilde{\kappa}_i$ is defined as:
\begin{equation}
    \nu_i = \frac{\kappa_i - \langle \kappa \rangle}{\sigma_\kappa},
    \label{eq:kappa_normalized}
\end{equation}
where $\langle \kappa \rangle$ is the mean convergence. Typically, the mean convergence is approximated to zero after mean-field subtraction:
\begin{equation}
    \langle \kappa \rangle = \frac{1}{N_{\mathrm{pix}}} \sum_{i=1}^{N_{\mathrm{pix}}} \kappa_i \approx 0.
\end{equation}
Normalization ensures that the PDF satisfies the standard probability normalization condition:
\begin{equation}
    \int_{-\infty}^{\infty} P(\nu) \, d\nu = 1,
    \label{eq:pdf_normalization}
\end{equation}


