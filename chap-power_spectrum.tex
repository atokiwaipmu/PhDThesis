The power spectrum constitutes a fundamental second-order statistical measure that characterizes the distribution of fluctuations across varying spatial scales. 
In this chapter, we provide an overview of the power spectrum, focusing on its definition, properties, and cosmological applications following \citet{2000PhRvD..62d3007H} and \citet{2017MNRAS.472.2126K}.

\section{Overview of Power Spectrum}
Power Spectrum is defined as the Fourier transform of the two-point correlation function, $\xi(\mathbf{x})$, which quantifies the statistical correlation between density fluctuations at two distinct spatial positions. Formally, the two-point correlation function is expressed as:
\begin{equation}
    \xi(\mathbf{x}) := \langle \delta(\mathbf{x}') \delta(\mathbf{x}' + \mathbf{x}) \rangle,
    \label{eq:correlation}
\end{equation}
where $\delta(\mathbf{x})$ represents the density fluctuation at position $\mathbf{x}$, and the angular brackets $\langle \cdot \rangle$ denote the ensemble average over all realizations of the density field.

Subsequently, the correlation of these fluctuations in Fourier space is considered. By taking the Fourier transform of the two-point correlation function, we obtain:
\begin{equation}
    \begin{split}
        \langle \tilde{\delta}(\mathbf{k}) \tilde{\delta}(\mathbf{k}') \rangle &= \int \mathrm{d}\mathbf{x}\, e^{-i\mathbf{k} \cdot \mathbf{x}} \int \mathrm{d}\mathbf{x}'\, e^{-i\mathbf{k}' \cdot \mathbf{x}'} \langle \delta(\mathbf{x}) \delta(\mathbf{x}') \rangle \\
        &= (2\pi)^3 \delta^{(3)}(\mathbf{k} + \mathbf{k}') \int \mathrm{d}\mathbf{x}\, e^{-i\mathbf{k} \cdot \mathbf{x}} \xi(\mathbf{x}),
    \end{split}
    \label{eq:power_spectrum_definition}
\end{equation}
where $\tilde{\delta}(\mathbf{k})$ denotes the Fourier transform of the density fluctuation field $\delta(\mathbf{x})$, and $\delta^{(3)}$ represents the three-dimensional Dirac delta function.

From this formulation, the power spectrum $P(k)$ is defined as:
\begin{equation}
    \langle \tilde{\delta}(\mathbf{k}) \tilde{\delta}(\mathbf{k}') \rangle := (2\pi)^3 \delta^{(3)}(\mathbf{k} + \mathbf{k}') P(k), \quad \text{where} \quad P(k) := \int \mathrm{d}\mathbf{x}\, e^{-i\mathbf{k} \cdot \mathbf{x}} \xi(\mathbf{x}).
    \label{eq:power_spectrum}
\end{equation}
Here, $P(k)$ encapsulates the amplitude of density fluctuations as a function of the wavenumber $k$.

This formalism can be naturally extended to two-dimensional (2D) analyses by substituting the three-dimensional wavenumber vector $\mathbf{k}$ with a two-dimensional counterpart, $\boldsymbol{\ell}$, and correspondingly replacing the three-dimensional power spectrum with the two-dimensional power spectrum, $C(\boldsymbol{\ell})$. The two-dimensional correlation function is defined analogously:
\begin{equation}
    \omega(\boldsymbol{\theta}) := \langle \delta(\boldsymbol{\theta}') \delta(\boldsymbol{\theta}' + \boldsymbol{\theta}) \rangle,
    \label{eq:2D_correlation}
\end{equation}
where $\boldsymbol{\theta}$ denotes the angular separation vector in the 2D plane.

Consequently, the two-dimensional power spectrum is expressed as:
\begin{equation}
    \langle \tilde{\delta}(\boldsymbol{\ell}) \tilde{\delta}(\boldsymbol{\ell}') \rangle := (2\pi)^2 \delta^{(2)}(\boldsymbol{\ell} + \boldsymbol{\ell}') C(\boldsymbol{\ell}),
    \label{eq:2D_power_spectrum}
\end{equation}
where $\delta^{(2)}$ is the two-dimensional Dirac delta function, and $C(\boldsymbol{\ell})$ is defined by the Fourier transform of the 2D correlation function:
\begin{equation}
    C(\boldsymbol{\ell}) = \int \mathrm{d}^2\boldsymbol{\theta}\, e^{-i\boldsymbol{\ell} \cdot \boldsymbol{\theta}} \omega(\boldsymbol{\theta}) = 2\pi \int \theta\, \mathrm{d}\theta\, J_0(\ell \theta) \omega(\theta).
    \label{eq:2D_power_spectrum_definition}
\end{equation}
In the final equality, the integral has been converted to polar coordinates, where $J_0$ denotes the zeroth-order Bessel function of the first kind, and $\ell = |\boldsymbol{\ell}|$ is the magnitude of the 2D wavenumber vector. 

\section{Two-Point Correlation Function}
The real-space shear two-point correlation function is one of the most basic observables for cosmic shear. The shear can be decomposed into a tangential $\gamma_+$ and a cross-component $\gamma_\times$ with respect to a reference point $\boldsymbol{\theta}$:
\begin{equation}
    \begin{split}
        \gamma_+(\boldsymbol{\theta}, \boldsymbol{\theta}') &= -\Re[\gamma(\boldsymbol{\theta}') e^{-2i\phi}], \\
        \gamma_\times(\boldsymbol{\theta}, \boldsymbol{\theta}') &= -\Im[\gamma(\boldsymbol{\theta}') e^{-2i\phi}],
    \end{split}
    \label{eq:shear_decomposition}
\end{equation}
where $\Re$ and $\Im$ denote the real and imaginary parts, respectively, and $\phi$ is the polar angle between $\boldsymbol{\theta}$ and $\boldsymbol{\theta}'$.
Following \citet{2015RPPh...78h6901K}, the two-point correlators $\langle \gamma_+ \gamma_+\rangle, \langle \gamma_\times \gamma_\times \rangle$ can be combined into the two components of the shear 2PCF:
\begin{equation}
    \begin{split}
        \xi_+(\theta) &= \langle \gamma \gamma^* \rangle (\boldsymbol{\theta}) = \langle \gamma_+ \gamma_+ \rangle (\boldsymbol{\theta}) + \langle \gamma_\times \gamma_\times \rangle (\boldsymbol{\theta}), \\
        \xi_\times(\theta) &= \Re \left[ \langle \gamma \gamma \rangle (\boldsymbol{\theta}) e^{-4i\phi} \right] = \langle \gamma_+ \gamma_- \rangle (\boldsymbol{\theta}) + \langle \gamma_\times \gamma_\times \rangle (\boldsymbol{\theta})
    \end{split}
    \label{eq:shear_2PCF}
\end{equation}
The main reason for this decomposition is that thay can be estimated from the measured ellipticities of galaxies $\epsilon_i$ as:
\begin{equation}
    \gamma_\pm (\boldsymbol{\theta}) = \frac{\sum_{i, j} w_i w_j \epsilon_{+, i}\epsilon_{+, j} \pm \epsilon_{\times, i}\epsilon_{\times, j}}{\sum_{i, j} w_i w_j},
    \label{eq:shear_estimator}
\end{equation}
where the sun runs over all the galaxy pairs and the weight $w_i$ of the ellipticity of the $i$-th galaxy can inblude the measurement error and the intrinsic ellipticity of the galaxy.
By going to the Fourier space and then back to the real space, the shear 2PCF can be expressed as:
\begin{eqnarray}
    \xi_+(\theta) &=& \frac{1}{2\pi} \int \left[P_\kappa^E(\ell) + P_\kappa^B(\ell)\right] J_0(\ell \theta) \ell \, \mathrm{d}\ell, \\
    \xi_\times(\theta) &=& \frac{1}{2\pi} \int \left[P_\kappa^E(\ell) - P_\kappa^B(\ell)\right] J_4(\ell \theta) \ell \, \mathrm{d}\ell,
\end{eqnarray}
where $P_\kappa^E$ and $P_\kappa^B$ are the E-mode and B-mode power spectra of the convergence field, respectively, and $J_n$ denotes the $n$-th order Bessel function of the first kind.
For the convergence case, the 2PCF can be expressed as:
\begin{equation}
    \xi(\theta) = \frac{1}{2\pi} \int P_\kappa(\ell) J_0(\ell \theta) \ell \, \mathrm{d}\ell.
\end{equation}
which is exactly the same as the shear 2PCF $\xi_+(\theta)$ as $P_\kappa^E(\ell) = P_\kappa(\ell)$.

\section{Power Spectrum}
\subsection{Matter Power Spectrum}
The matter power spectrum, \( P(k) \), is a Fourier transform of the two-point correlation function of the dark matter density field, \( \delta(\mathbf{x}) \).
The matter power spectrum is defined through the relation:
\begin{equation}
    \langle \tilde{\delta}(\mathbf{k}) \tilde{\delta}(\mathbf{k}') \rangle = (2\pi)^3 \delta^{(3)}(\mathbf{k} + \mathbf{k}') P(k),
    \label{eq:matter_power_spectrum}
\end{equation}
\subsection{Convergence Power Spectrum}
However, in the context of weak lensing, the matter power spectrum is the quantity that is not directly observable. Instead, the angular power spectrum of the convergence field, \( C_{\ell}^{\kappa\kappa} \), is the quantity that is directly observable.
The convergence power spectrum, \( P_{\kappa}(\ell) \), quantifies the statistical properties of \( \kappa(\boldsymbol{\theta}) \) in Fourier space and is defined as:
\begin{equation}
    \langle \tilde{\kappa}(\boldsymbol{\ell}) \tilde{\kappa}(\boldsymbol{\ell}') \rangle = (2\pi)^2 \delta^{(2)}(\boldsymbol{\ell} + \boldsymbol{\ell}') P_{\kappa}(\ell),
    \label{eq:convergence_power_spectrum}
\end{equation}
As we have seen in Eq.~\eqref{eq:convergence_integral}, the convergence field \( \kappa(\boldsymbol{\theta}) \) can be expressed as a weighted projection of the matter density contrast along the line of sight:
\begin{equation}
    \kappa(\boldsymbol{\theta}) = \int_0^{\chi_s} d\chi \, W(\chi) \, \delta_m\left(\chi \boldsymbol{\theta}, \chi\right),
\end{equation}
Recognizing the Fourier transform of the density field \( \delta_m\left(\chi \boldsymbol{\theta}, \chi\right) \):
\begin{equation}
    \delta_m\left(\chi \boldsymbol{\theta}, \chi\right) = \int \frac{d^3\mathbf{k}}{(2\pi)^3} \tilde{\delta}(\mathbf{k}) e^{i \mathbf{k} \cdot \mathbf{x}},
\end{equation}
we substitute into the Fourier transform of the convergence field:
\begin{eqnarray}
    \tilde{\kappa}(\boldsymbol{\ell}) &=& \int_0^{\chi_s} d\chi \, W(\chi) \int d\boldsymbol{\theta} \, e^{-i \boldsymbol{\ell} \cdot \boldsymbol{\theta}} \delta\left(\chi \boldsymbol{\theta}, \chi\right) \nonumber \\
    &=& \int_0^{\chi_s} d\chi \, W(\chi) \int d\boldsymbol{\theta} \, e^{-i \boldsymbol{\ell} \cdot \boldsymbol{\theta}} \int \frac{d^3\mathbf{k}}{(2\pi)^3} \tilde{\delta}(\mathbf{k}) e^{i \mathbf{k} \cdot \mathbf{x}} \nonumber \\
    &=& \int_0^{\chi_s} d\chi \, W(\chi) \int d\boldsymbol{\theta} \, e^{-i \boldsymbol{\ell} \cdot \boldsymbol{\theta}} \int \frac{dk_\parallel}{2\pi}\frac{d\mathbf{k}_\perp}{(2\pi)^2} \tilde{\delta}_m(k_\parallel, \mathbf{k}_\perp) e^{i (k_\parallel \chi  + \chi \mathbf{k}_\perp \cdot \boldsymbol{\theta})} \nonumber \\
    &=& \int_0^{\chi_s} d\chi \, \frac{W(\chi)}{\chi^2} \int \frac{dk_\parallel}{2\pi} \tilde{\delta}_m\left(k_\parallel, \frac{\boldsymbol{\ell}}{\chi}\right) e^{i k_\parallel \chi}
    \label{eq:kappa_fourier_final}
\end{eqnarray}
where \( \mathbf{k}_\perp \) is the component of \( \mathbf{k} \) perpendicular to the line of sight, and \( k_\parallel \) is the component along the line of sight.
Then the ensemble average of the Fourier transform of the convergence field is:
\begin{eqnarray}
    \langle \tilde{\kappa}(\boldsymbol{\ell}) \tilde{\kappa}(\boldsymbol{\ell}') \rangle &=& \int_0^{\chi_s} d\chi \, \frac{W(\chi)}{\chi^2} \int_0^{\chi_s} d\chi' \, \frac{W(\chi')}{\chi'^2} \int \frac{dk_\parallel}{2\pi} \int \frac{dk'_\parallel}{2\pi} \langle \tilde{\delta}_m\left(k_\parallel, \frac{\boldsymbol{\ell}}{\chi}\right) \tilde{\delta}_m\left(k'_\parallel, \frac{\boldsymbol{\ell}'}{\chi'}\right) \rangle e^{i k_\parallel \chi} e^{i k'_\parallel \chi'} \nonumber \\
    &=& \int_0^{\chi_s} d\chi \, \frac{W(\chi)}{\chi^2} \int_0^{\chi_s} d\chi' \, \frac{W(\chi')}{\chi'^2} \int \frac{dk_\parallel}{2\pi} e^{i k_\parallel (\chi - \chi')} (2\pi)^2 \delta^{(2)}\left(\frac{\boldsymbol{\ell}}{\chi} + \frac{\boldsymbol{\ell}'}{\chi'}\right) P_m\left(\sqrt{k_\parallel^2 + \frac{\ell^2}{\chi^2}}\right) \nonumber
\end{eqnarray}
Under Limber approximation ($k_\parallel \ll \ell/\chi$; \citealp{1954ApJ...119..655L}), the last integral can be approximated as:
\begin{equation}
    \int \frac{dk_\parallel}{2\pi} e^{i k_\parallel (\chi - \chi')} P_m\left(\sqrt{k_\parallel^2 + \frac{\ell^2}{\chi^2}}\right) \approx P_m\left(\frac{\ell}{\chi}\right) \int \frac{dk_\parallel}{2\pi} e^{i k_\parallel (\chi - \chi')} = P_m\left(\frac{\ell}{\chi}\right) \delta(\chi - \chi')
\end{equation}
Therefore, we obtain the simplified ensemble average of the Fourier transform of the convergence field:
\begin{equation}
    \langle \tilde{\kappa}(\boldsymbol{\ell}) \tilde{\kappa}(\boldsymbol{\ell}') \rangle = (2\pi)^2 \delta^{(2)}(\boldsymbol{\ell} + \boldsymbol{\ell}') \int_0^{\chi_s} d\chi \, \frac{W^2(\chi)}{\chi^2} P_m\left(\frac{\ell}{\chi}; \chi\right)
    \label{eq:kappa_power_spectrum}
\end{equation}
Thereby, substituting this into the definition of the convergence power spectrum (Eq.~\eqref{eq:convergence_power_spectrum}), we get:
\begin{equation}
    C_{\ell}^{\kappa\kappa} = \int_0^{\chi_s} d\chi \, \frac{W^2(\chi)}{\chi^2} P_m\left(\frac{\ell}{\chi}; \chi\right)
    \label{eq:convergence_power_spectrum_final}
\end{equation}

\section{Estimator}
In practice, the power spectrum is estimated from the observed data. The angular power spectrum is estimated by averaging over the $2\ell + 1$ $m$-modes for each $\ell$:
\begin{equation}
    \hat{C}_\ell = \frac{1}{2\ell + 1} \sum_{m=-\ell}^{\ell} |a_{\ell m}|^2.
\end{equation}
where convergence field is expanded in spherical harmonics:
\begin{equation}
    \kappa(\boldsymbol{\theta}) = \sum_{\ell, m} a_{\ell m} Y_{\ell m}(\boldsymbol{\theta}).
    \label{eq:kappa_expansion}
\end{equation}
This estimator provides an unbiased estimate of the true angular power spectrum under the assumption of statistical isotropy.


