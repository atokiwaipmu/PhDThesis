Numerical simulations are crucial in physics and astronomy for studying complex systems of interacting particles, such as galaxies and the Universe's large-scale structure, where analytical solutions are often impractical due to complexity and nonlinearity. This section provides an overview of $N$-body simulations commonly used in cosmology.

Since the 1980s, numerical cosmology has developed algorithms to mitigate the computational challenges posed by long-range gravitational interactions by reducing global communication across the computational domain. Key algorithms include mesh-based methods, tree codes, and multipole expansions~\citep{1981csup.book.....H}. Figure~\ref{fig:particle-count} displays the number of particles used in selected $N$-body simulations employing these techniques. Symbols and colors indicate the gravitational solvers: particle-particle-particle-mesh (P$^3$M) and adaptive P$^3$M (AP$^3$M); parallel or vectorized P$^3$M; Tree codes; TreePM; and particle-mesh methods with adaptive mesh refinement (PM AMR).

Advancements in algorithms and software optimization have increased the number of particles in cosmological simulations beyond what direct summation methods allow. Since 1990, gravity-only simulations have exhibited a super-exponential growth trend, indicated by the quadratic regression in Figure~\ref{fig:particle-count}, reflecting significant methodological innovations beyond hardware improvements~\citep{leclercq2020}.

\begin{figure} \centering \includegraphics[width=0.6\textwidth]{figures/Moore_law_cosmosims.png} \caption{Evolution of the number of particles used in $N$-body simulations as a function of the year of publication~\citep{leclercq2020}. The symbols and colors indicate the gravitational solver employed: P$^3$M and adaptive P$^3$M (AP$^3$M); parallel or vectorized P$^3$M; Tree codes; TreePM; and particle-mesh methods with adaptive mesh refinement (PM AMR). Hydrodynamic simulations are represented by black squares.} \label{fig:particle-count} \end{figure}

\section{Initial Condition Generation}
As we have seen in Section~\ref{sec:initial_conditions}, the primodial power spectrum $P(k)$ is a key ingredient in generating initial conditions for cosmological simulations. Based on the linear power spectrum, we will review the process of generating initial conditions for $N$-body simulations.

\subsection{Initial Density Field}
To generate the initial density field for the simulations, we express the density contrast $\delta(\mathbf{x})$ in terms of its Fourier components:
\begin{equation}
    \delta(\mathbf{x}) = \int \frac{d^3k}{(2\pi)^3} \tilde{\delta}(\mathbf{k}) e^{i\mathbf{k} \cdot \mathbf{x}}.
\end{equation}
Assuming a Gaussian random field, each Fourier mode $\tilde{\delta}(\mathbf{k})$ is a complex Gaussian random variable with zero mean and variance $P(k)$:
\begin{align}
    \tilde{\delta}(\mathbf{k}) &= A(\mathbf{k}) + i B(\mathbf{k}), \\
    \langle A(\mathbf{k}) \rangle &= \langle B(\mathbf{k}) \rangle = 0, \\
    \langle A(\mathbf{k}) A(\mathbf{k}') \rangle &= \langle B(\mathbf{k}) B(\mathbf{k}') \rangle = \frac{1}{2} P(k) \delta_D(\mathbf{k} - \mathbf{k}'), \\
    \langle A(\mathbf{k}) B(\mathbf{k}') \rangle &= 0,
\end{align}
where $A(\mathbf{k})$ and $B(\mathbf{k})$ are real Gaussian random variables, and $\delta_D$ is the Dirac delta function.

\subsection{Initial Displacement Field}
The initial displacement field $\boldsymbol{\Psi}(\mathbf{q})$ relates the Lagrangian coordinates $\mathbf{q}$ to the Eulerian coordinates $\mathbf{x}$:
\begin{equation}
    \mathbf{x}(\mathbf{q}) = \mathbf{q} + \boldsymbol{\Psi}(\mathbf{q}).
\end{equation}
The displacement field is proportional to the gradient of the gravitational potential $\Phi(\mathbf{q})$:
\begin{equation}
    \boldsymbol{\Psi}(\mathbf{q}) = - \nabla \Phi(\mathbf{q}),
\end{equation}
where the potential satisfies Poisson's equation:
\begin{equation}
    \nabla^2 \Phi(\mathbf{q}) = \delta(\mathbf{q}).
\end{equation}
The first order solution to the displacement field is given by the Zel'dovich approximation \citep{1970A&A.....5...84Z}:
\begin{align}
    -k^2 \tilde{\Phi}(\mathbf{k}) &= \tilde{\delta}(\mathbf{k}), \\
    \tilde{\boldsymbol{\Psi}}(\mathbf{k}) &= i \mathbf{k} \tilde{\Phi}(\mathbf{k}) = i \mathbf{k} \frac{\tilde{\delta}(\mathbf{k})}{k^2}, \\
    \boldsymbol{\Psi}(\mathbf{q}) &= \int \frac{d^3k}{(2\pi)^3} i \mathbf{k} \frac{\tilde{\delta}(\mathbf{k})}{k^2} e^{i\mathbf{k} \cdot \mathbf{q}}.
\end{align}

\subsection{Initial Velocities}
The initial velocities of particles are derived from the time derivative of the displacement field. The velocities are given by \citep{1985ApJS...57..241E}:
\begin{align}
    \mathbf{v}(\mathbf{q}) &= a H f(a) \boldsymbol{\Psi}(\mathbf{q}), \\
    \tilde{\mathbf{v}}(\mathbf{k}) &= a H f(a) \tilde{\boldsymbol{\Psi}}(\mathbf{k}) = a H f(a) i \mathbf{k} \frac{\tilde{\delta}(\mathbf{k})}{k^2}, \\
    \mathbf{v}(\mathbf{q}) &= i a H f(a) \int \frac{d^3k}{(2\pi)^3} \frac{\mathbf{k}}{k^2} \tilde{\delta}(\mathbf{k}) e^{i\mathbf{k} \cdot \mathbf{q}},
\end{align}
where $a$ is the scale factor, $H$ is the Hubble parameter, and $f(a)$ is the growth rate defined as:
\begin{equation}
    f(a) = \frac{d\ln D}{d\ln a},
\end{equation}

\section{Simulation Basics}
We outline the fundamental concepts and algorithms used in $N$-body simulations, including direct summation, particle-mesh methods, particle-particle particle-mesh (P3M) methods, and tree-particle-mesh (Tree-PM) methods.

\subsection{Direct Summation}
Direct Summation calculates gravitational forces between all particle pairs directly, scaling as $\mathcal{O}(N^2)$ and becoming computationally intensive for large $N$. Each particle $i$ has position $\mathbf{r}_i$, velocity $\mathbf{v}_i$, and mass $m_i$. At each time step $t$:
\begin{enumerate}
    \item \textbf{Compute Forces:}
    \[
    \mathbf{F}_i = G m_i \sum_{\substack{j=1 \\ j \neq i}}^{N} \frac{m_j (\mathbf{r}_j - \mathbf{r}_i)}{\|\mathbf{r}_j - \mathbf{r}_i\|^3}
    \]
    
    \item \textbf{Update Particle States:}
    \[
    \mathbf{v}_i(t + \Delta t) = \mathbf{v}_i(t) + \frac{\mathbf{F}_i}{m_i} \Delta t, \quad \mathbf{r}_i(t + \Delta t) = \mathbf{r}_i(t) + \mathbf{v}_i(t + \Delta t) \Delta t
    \]
    
    \item \textbf{Advance Time:}
    \[
    t \leftarrow t + \Delta t
    \]
\end{enumerate}

\subsection{Particle-Mesh (PM) Method} \label{sec:pm-method}
The PM method approximates gravitational forces by mapping particles onto a grid and solving for the gravitational potential, reducing computational cost to $\mathcal{O}(N + M \log M)$, where $M$ is the number of grid points. While efficient for large-scale simulations, it smooths out small-scale forces.

The main difference between the PM method and direct summation is the grid-based force calculation:
\begin{enumerate}
    \item \textbf{Assign Particles to Grid:} See Section~\ref{sec:mass-assignment}.
    \item \textbf{Compute Density Field:}
    \[
    \rho(\mathbf{x}) = \sum_i m_i W(\mathbf{x} - \mathbf{r}_i) \quad \text{(where $W$: Interpolation Kernel)}
    \]
    \item \textbf{Solve Poisson's Equation:}
    \[
    \nabla^2 \Phi = 4\pi G \rho
    \]
    \item \textbf{Compute Force due to Potential:}
    \[
    \mathbf{E} = -\nabla \Phi
    \]
\end{enumerate}

\subsection{Particle-Particle Particle-Mesh (P3M) Method}
The P$^3$M method combines direct summation for short-range forces with the PM approach for long-range interactions, achieving $\mathcal{O}(N \log N)$ complexity while enhancing accuracy for nearby particles. Key parameters include mesh size, softening parameter $\epsilon$, and force resolution.

The difference between the P$^3$M method and the PM method lies in the force calculation:
\begin{enumerate}
    \item \textbf{Long-Range Forces (PM):}
    \[
    \mathbf{F}_{\text{long},i} = m_i \mathbf{E}_{\text{long}}(\mathbf{r}_i)
    \]
    \item \textbf{Short-Range Forces (Direct Summation):}
    \begin{enumerate}[label={(\alph*)}]
        \item \textbf{Neighbor Search:}
        Identify particles $j$ within cutoff radius $r_{\text{cut}}$ of particle $i$.
        \item \textbf{Force Calculation:}
        \[
        \mathbf{F}_{\text{short},i} = -G m_i \sum_{\substack{j \in \text{neighbors}}} \frac{m_j (\mathbf{r}_i - \mathbf{r}_j)}{\left(\|\mathbf{r}_i - \mathbf{r}_j\|^2 + \epsilon^2\right)^{3/2}}
        \]
    \end{enumerate}
    \item \textbf{Combine Forces:}
    \[
    \mathbf{F}_i = \mathbf{F}_{\text{long},i} + \mathbf{F}_{\text{short},i}
    \]
\end{enumerate}

\subsection{Tree-Particle-Mesh (Tree-PM) Method}
The Tree-PM method integrates the PM approach for long-range forces with a tree algorithm for short-range interactions, reducing complexity to $\mathcal{O}(N \log N)$. Proper tuning of parameters like grid size, softening length $\epsilon$, and opening angle $\theta_{\text{max}}$ is essential.

The main updates in the Tree-PM method compared to the P$^3$M method are in the tree construction when calculating short-range forces:
\begin{enumerate}
    \item \textbf{Tree Construction:}
    \begin{enumerate}[label={(\alph*)}]
        \item \textbf{Build Spatial Cells:}
        Partition the simulation volume into spatial cells (e.g., octree) and assign particles to nodes.
        \item \textbf{Multipole Moments:}
        For each node $j$, calculate mass $M_j$ and center of mass $\mathbf{r}_{\text{cm},j}$.
    \end{enumerate}
    \item \textbf{Force Calculation:}
    For each particle $i$, traverse the tree to compute the short-range gravitational force:
        \[
        \mathbf{F}_{\text{short},i} = -G m_i \sum_{\text{nodes}} \frac{M_j (\mathbf{r}_i - \mathbf{r}_j)}{(\|\mathbf{r}_i - \mathbf{r}_j\|^2 + \epsilon^2)^{3/2}}
        \]
        using the opening angle criterion:
        \[
        \theta = \frac{l_j}{\|\mathbf{r}_i - \mathbf{r}_j\|} < \theta_{\text{max}}
        \]
        where $l_j$ is the size of node $j$ and $\theta_{\text{max}}$ is the maximum allowed opening angle.
\end{enumerate}

\section{Tools for Fast Computation}
Efficient computational tools are crucial for large-scale simulations and data analysis in scientific and engineering applications. This section overviews key computational techniques and algorithms used in $N$-body simulations and large-scale structure studies.

\subsection{Fast Fourier Transform}
The Fast Fourier Transform (FFT) is a highly efficient algorithm for computing the Discrete Fourier Transform (DFT) of a sequence. Given a sequence of $N$ complex numbers $\{x_n\}_{n=0}^{N-1}$, the DFT is defined as:
\begin{equation}
    X_k = \sum_{n=0}^{N-1} x_n e^{-2\pi i kn / N}, \quad k = 0, 1, \dots, N-1.
\end{equation}
The naive computation of the DFT requires $\mathcal{O}(N^2)$ operations. The FFT reduces this complexity to $\mathcal{O}(N \log N)$ by exploiting the symmetry and periodicity properties of the exponential kernel. The most common FFT algorithm is the Cooley-Tukey radix-2 FFT \citep{d3ea2d52-5ab2-3128-8b80-efb85267295d}, which recursively decomposes the DFT into smaller DFTs of even and odd-indexed elements:
\begin{align}
    X_k &= \sum_{n=0}^{N/2-1} x_{2n} e^{-2\pi i k (2n) / N} + \sum_{n=0}^{N/2-1} x_{2n+1} e^{-2\pi i k (2n+1) / N} \\
         &= X_k^{\text{even}} + e^{-2\pi i k / N} X_k^{\text{odd}},
\end{align}
where $X_k^{\text{even}}$ and $X_k^{\text{odd}}$ are the DFTs of the even and odd subsequences, respectively.

\subsection{Mass Assignment Schemes}\label{sec:mass-assignment}
Mass assignment schemes map particle masses onto a computational grid to compute density fields and gravitational forces, ensuring mass conservation and minimizing aliasing errors. Common schemes include:
\begin{itemize}
    \item \textbf{Nearest Grid Point (NGP):} Each particle is assigned entirely to the nearest grid point.
    \item \textbf{Cloud-In-Cell (CIC):} Mass is linearly interpolated to the nearest $2^3 = 8$ surrounding grid points.
    \item \textbf{Triangular-Shaped Cloud (TSC):} Mass is distributed to the nearest $3^3 = 27$ grid points using a quadratic interpolation function.
\end{itemize}
In Fourier space, these mass assignment window functions are represented as:
\begin{equation}
    W(\mathbf{k}) = \prod_{i=1}^{3} W(k_i),
\end{equation}
where
\begin{equation}
    W(k_i) = \left[\frac{\sin\left(\pi k_i / (2 k_N)\right)}{\pi k_i / (2 k_N)}\right]^p,
\end{equation}
with $k_N$ being the Nyquist wavenumber, $k_i$ the $i$-th component of the wavevector $\mathbf{k}$, and $p = 1$ for NGP, $p = 2$ for CIC, and $p = 3$ for TSC \citep{1981csup.book.....H, 1985ApJS...57..241E}.

Figure~\ref{fig:mass-assignment} illustrates the mass assignment process for a particle distribution on a 1D grid using different schemes.
\begin{figure}[ht]
    \centering
    \includegraphics[width=\textwidth]{figures/weight_functions.png}
    \caption{Illustration of three mass assignment schemes—Nearest Grid Point (NGP), Cloud-In-Cell (CIC), and Triangular-Shaped Cloud (TSC)—used to map a particle's mass onto a 1D grid.}
    \label{fig:mass-assignment}
\end{figure}

\subsection{Parallelization Techniques}
Parallelization accelerates computations in large-scale simulations by leveraging multiple processors or computing nodes. Key strategies include:
\begin{itemize}
    \item \textbf{Domain Decomposition:} The computational domain is partitioned into smaller subdomains, each assigned to a separate processor \citep{1986Natur.324..446B}.
    \item \textbf{Task Parallelism:} Distributing independent tasks across multiple processors.
    \item \textbf{Data Parallelism:} Performing identical operations concurrently on different data elements, enabling SIMD (Single Instruction, Multiple Data) execution.
\end{itemize}

\subsection{Adaptive Mesh Refinement}
Adaptive Mesh Refinement (AMR) dynamically adjusts grid resolution, refining the mesh where higher accuracy is needed (e.g., regions with high density gradients) and coarsening it elsewhere \citep{1989JCoPh..82...64B}. This creates a hierarchy of grids with increasing resolution and optimizes computational resources. Refinement is typically triggered when: 
\begin{equation}
    \left| \nabla \phi(\mathbf{x}) \right| > \theta,
\end{equation}
with $\theta$ being a predefined threshold.

\begin{figure}
    \centering
    \includegraphics[width=\textwidth]{figures/adaptive_mesh_refinement.png}
    \caption{Illustration of adaptive mesh refinement (AMR) applied to a 2D image with two Gaussian kernels. The left panel shows the initial coarse grid structure over the image. The middle and right panels demonstrate progressively finer levels of mesh refinement in regions of higher intensity, where the Gaussian kernels are located. The red grid outlines indicate the adaptively refined mesh hierarchy, ensuring higher resolution where needed while maintaining computational efficiency in lower-intensity regions.}
    \label{fig:amr}
\end{figure}
Figure~\ref{fig:amr} demonstrates the application of Adaptive Mesh Refinement (AMR) to a two-dimensional image containing two Gaussian kernels. Initially, a uniformly coarse grid overlays the entire image (left panel). As the refinement process progresses, the mesh becomes increasingly finer in regions with higher intensity, specifically around the Gaussian kernels (middle and right panels). The red grid lines represent the hierarchy of the refined meshes, enabling higher resolution where it is most needed and optimizing computational resources by keeping a coarser grid in less significant areas.

\subsection{Tree Construction}
Tree-based data structures efficiently organize hierarchical spatial data. The Barnes-Hut algorithm employs an octree to partition space, reducing computational complexity from $\mathcal{O}(N^2)$ to $\mathcal{O}(N \log N)$ by approximating distant particle clusters as single mass points. This approximation is controlled by the opening angle $\theta$:
\begin{equation}
    \frac{s}{d} < \theta,
\end{equation}
where $s$ is the node size and $d$ is the distance from the particle to the node's center of mass.

One of the popular algorithms for tree construction is the Barnes-Hut Octree \citep{1986Natur.324..446B}, which recursively subdivides the simulation volume into hierarchical grid cells. Figure~\ref{fig:barnes-hut} illustrates the Octree decomposition for a 3D volume containing four particles.
\begin{figure}[ht]
    \centering
    \includegraphics[width=0.6\textwidth]{figures/Octree.png}
    \caption{Illustration of an Octree decomposition for a 3D volume containing four particles. The left panel showcases the spatial subdivision of the volume into hierarchical grid cells, with color-coding indicating different levels of refinement. The right panel presents the corresponding Octree data structure, highlighting the hierarchical relationships between nodes. Credit by \citet{Powell2023}}
    \label{fig:barnes-hut}
\end{figure}

Parallel tree construction involves building local trees within each subdomain and integrating them for global computations \citep{DUBINSKI1996133}. Efficient parallelization enhances scalability and performance in large-scale simulations.

\section{\texttt{FASTPM}} \label{sec:fastpm}
\texttt{FASTPM} (Fast Particle Mesh; \citealt{10.1093/mnras/stw2123}) is an advanced N-body simulation code tailored for efficiently modeling the evolution of dark matter and halo structures on cosmological scales. Building upon the foundational PM approach, \texttt{FASTPM} integrates modified kick and drift factors derived from the Zel'dovich Approximation (ZA). This enhancement allows \texttt{FASTPM} to achieve high accuracy in large-scale structure formation while significantly reducing computational overhead. This subsection delineates the core methodology of FASTPM, incorporating the mathematical formalism of its modified kick and drift factors.

\subsection{Modified Kick and Drift Factors}
The cornerstone of \texttt{FASTPM}'s enhanced performance lies in its \textbf{modified kick ($K_{\text{FASTPM}}$)} and \textbf{drift ($D_{\text{FASTPM}}$)} factors. These factors are meticulously derived from the Zel'dovich Approximation (ZA), a first-order Lagrangian perturbation theory (1LPT), to rectify inaccuracies in large-scale growth inherent in standard PM solvers, especially when operating with a limited number of time steps.

First, the Zel'dovich equation of motion to the first order is defined as:
\begin{eqnarray}
    \mathbf{x}_{\text{ZA}}(a) &=& \mathbf{q} + D(a)\mathbf{s}_1,  \nonumber \\
    \mathbf{p}_{\text{ZA}}(a) &=& a^3 E(a) g_p(a) \mathbf{s}_1,  \nonumber \\
    \mathbf{f}_{\text{ZA}}(a) &=& a^2 E(a) g_f(a) \mathbf{s}_1, 
\end{eqnarray}
where $E(a) = \frac{H(a)}{H(a=1)}$ is the dimensionless Hubble parameter, and $g_p(a)$ and $g_f(a)$ are auxiliary factors defined as:
\begin{eqnarray}
    g_p(a)  &=& \frac{dD}{da},  G_p(a) = D(a) \\[0.5em]
    g_f(a)  &=& \frac{d(a^3 E g_p)}{da}, G_f(a) = a^3 E g_p(a)
\end{eqnarray}
The ZA equations of motion are reformulated in terms of drift and kick operators by integrating over a time step from $a_0$ to $a_1$ and eliminating the ZA displacement $\mathbf{s}_1$:
\begin{eqnarray}
    \Delta \mathbf{x}_{\text{ZA}} &=& \mathbf{x}_{\text{ZA}}(a_1) - \mathbf{x}_{\text{ZA}}(a_0) \nonumber \\
    &=& \left[ D(a) \right]_{a_0}^{a_1} \mathbf{s}_1 \nonumber \\
    &=& \frac{\mathbf{p}_{\text{ZA}}(a_r)}{a_r^3 E(a_r)} \left( \frac{\Delta G_p}{g_p(a_r)} \right),
\end{eqnarray}
\begin{eqnarray}
    \Delta \mathbf{p}_{\text{ZA}} &=& \mathbf{p}_{\text{ZA}}(a_1) - \mathbf{p}_{\text{ZA}}(a_0) \nonumber \\
    &=& \frac{\mathbf{f}_{\text{ZA}}(a_r)}{a_r^2 E(a_r)} \left( \frac{\Delta G_f}{g_f(a_r)} \right),
\end{eqnarray}
where $\Delta \mathbf{x}_{\text{ZA}}$ is the change in displacement over the time step, $\Delta \mathbf{p}_{\text{ZA}}$ is the change in momentum over the time step, $a_r$ is a reference scale factor within the time step, $\Delta G_p = G_p(a_1) - G_p(a_0)$, and $\Delta G_f = G_f(a_1) - G_f(a_0)$.
Therefore, the modified kick and drift factors in FASTPM are defined as:
\begin{eqnarray}
    \mathcal{D}_{\text{FASTPM}} &=& \frac{\Delta \mathbf{x}_{\text{ZA}}}{\mathbf{p}_{\text{ZA}}} 
        = \frac{1}{a_r^3 E(a_r)} \left( \frac{\Delta G_p}{g_p(a_r)} \right) \\[1em]
    \mathcal{K}_{\text{FASTPM}} &=& \frac{\Delta \mathbf{p}_{\text{ZA}}}{\mathbf{f}_{\text{ZA}}} 
        = \frac{1}{a_r^2 E(a_r)} \left( \frac{\Delta G_f}{g_f(a_r)} \right) 
\end{eqnarray}
These operators ensure the exact integration of the ZA equations of motion, thereby accurately capturing the linear growth of structures within each time step.

\subsection{Algorithm Steps}
The main steps of the \texttt{FASTPM} algorithm follow the standard PM method discussed in Section~\ref{sec:pm-method}, with the addition of the modified kick and drift operators to ensure accurate linear growth.

\begin{enumerate}
    \item \textbf{Apply Modified Operators:}
    \label{fastpm:modified-kick-drift}
    Utilize the modified kick ($K_{\text{FASTPM}}$) and drift ($D_{\text{FASTPM}}$) factors to update particle velocities and positions. These factors, derived from the ZA, ensure accurate linear growth:
    
    \begin{enumerate}
        \item \textbf{Kick Step:}
        Update particle velocities by applying the gravitational acceleration scaled by the modified kick factor:
        \[
        \mathbf{v}_i\left(t + \frac{\Delta t}{2}\right) = \mathbf{v}_i(t) + \mathbf{g}_i(t) \cdot K_{\text{FASTPM}} \cdot \Delta t
        \]
        
        \item \textbf{Drift Step:}
        Update particle positions using the updated velocities and the modified drift factor:
        \[
        \mathbf{r}_i(t + \Delta t) = \mathbf{r}_i(t) + \mathbf{v}_i\left(t + \frac{\Delta t}{2}\right) \cdot D_{\text{FASTPM}} \cdot \Delta t
        \]
        
        \item \textbf{Second Kick Step:}
        Apply another kick to update velocities to the full time step:
        \[
        \mathbf{v}_i(t + \Delta t) = \mathbf{v}_i\left(t + \frac{\Delta t}{2}\right) + \mathbf{g}_i(t + \Delta t) \cdot K_{\text{FASTPM}} \cdot \Delta t
        \]
    \end{enumerate}
    
    \item \textbf{Update Particle States:}
    Finalize the update of particle velocities and positions after applying the modified kick and drift operators:
    \[
    \mathbf{v}_i(t + \Delta t) = \mathbf{v}_i\left(t + \frac{\Delta t}{2}\right) + \mathbf{g}_i(t + \Delta t) \cdot K_{\text{FASTPM}} \cdot \Delta t
    \]
    \[
    \mathbf{r}_i(t + \Delta t) = \mathbf{r}_i(t) + \mathbf{v}_i\left(t + \frac{\Delta t}{2}\right) \cdot D_{\text{FASTPM}} \cdot \Delta t
    \]
    
    \item \textbf{Advance Time:}
    Increment the simulation time by the time step $\Delta t$:
    \[
    t \leftarrow t + \Delta t
    \]
\end{enumerate}

\section{Weak Lensing Map Generation}
\label{sec:weak-lensing-generation}
\subsection{Ray-Tracing Algorithm}
Weak gravitational lensing distorts the shapes of background galaxies, providing a powerful probe of the large-scale structure of the Universe. The ray-tracing algorithm simulates the lensing effect by tracing light rays from background galaxies through the gravitational potential of the foreground matter distribution. The algorithm consists of the following steps:

\subsection{}

For each shell, the three-dimensional matter density $\delta(\mathbf{x}, z_i)$ was projected onto a two-dimensional plane perpendicular to the line of sight. The projected surface density $\Sigma(\hat{\mathbf{n}}, \chi_i)$ at an angular position $\hat{\mathbf{n}}$ was computed by integrating the matter density within the shell along the radial direction:
\begin{equation}
    \Sigma(\hat{\mathbf{n}}, \chi_i) = \int_{\chi_i}^{\chi_{i+1}} \delta(\chi \hat{\mathbf{n}}, z(\chi)) \, d\chi.
\end{equation}
In practice, the surface density was mapped onto a HEALPix grid \citep{Górski_2005} to create a full-sky map $\Sigma(n_j, \chi_i)$, where $n_j$ represents discretized angular positions. The HEALPix grid resolution was set to $N_{\text{side}} = 8192$, providing an angular resolution of approximately $0.43$ arcminutes, which is sufficient to capture small-scale structures relevant to weak lensing studies.

The convergence $\kappa(n_j; z_s)$ at each pixel of the HEALPix grid was then obtained by summing contributions from all the shells up to the source redshift:
\begin{equation}
    \kappa(n_j; z_s) = \sum_{i} W(\chi_i, \chi_s) \Sigma(n_j, \chi_i) \Delta \chi_i,
\end{equation}
where $W(\chi, \chi_s)$ is the lensing efficiency function.
The effective comoving distance to the center of the $j$-th shell, $\chi^j$, is calculated as \citep{2015MNRAS.453.3043S}:
\begin{equation}
    \chi^j = \frac{\int_{\chi_{\min}}^{\chi_{\max}} \chi^3 \, \mathrm{d}\chi}{\int_{\chi_{\min}}^{\chi_{\max}} \chi^2 \, \mathrm{d}\chi} = \frac{3}{4} \frac{\chi_{\max}^4 - \chi_{\min}^4}{\chi_{\max}^3 - \chi_{\min}^3},
\end{equation}
The density contrast within the $i$-th pixel of the $j$-th shell, $\delta^j(\hat{\boldsymbol{n}}_i)$, is determined by:
\begin{equation}
    \delta^j(\hat{\boldsymbol{n}}_i) = \frac{n_{\mathrm{part}, i}^j}{\bar{n}_{\mathrm{part}}^j} - 1,
\end{equation}
where $n_{\mathrm{part}, i}^j$ is the number of particles in the $i$-th pixel of the $j$-th shell, and $\bar{n}_{\mathrm{part}}^j$ is the average number of particles per pixel in that shell.