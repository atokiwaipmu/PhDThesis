In this study, we performed to set of simulations: TILED and BIGBOX to investigate the impacts of super-sample covariance on the covariance and correlation matrices of power spectrum and higher-order statistics. We utilize 11 realizations of state-of-art simulations, using $6144^3$ particles within 3750 Mpc/h, from the HalfDome project \citep{2024arXiv240717462B} as BIGBOX, and accordingly computed 20 realizations of the same volume tiled with smaller boxes with same cosmology, where each box component using $1024^3$ particles within $625$ Mpc/h, as TILED. We considered the convergence field which propagated under Born approximation. We extracted $10^\circ \times 10^\circ$ patches from full-sky convergence maps via a Fibonacci grid, we achieved uniform coverage with minimal overlap. 

A suite of higher-order statistics, including the angular power spectrum, bispectrum, probability distribution function (PDF), peak and minima counts, and Minkowski functionals, was employed to probe the non-Gaussian features of the convergence field. These measures were  computed using established tools and custom implementations, ensuring alignment with theoretical models like \texttt{Halofit}, \texttt{BiHalofit}, and \texttt{hmpdf}. By analyzing a broad range of scales, from $\ell$-binned statistics ($\ell \in [300,3000]$) to $\nu$-binned statistics ($-4 \leq \kappa/\sigma_\kappa \leq 4$), we provided a detailed characterization of the behavior of super-sample covariance.

We now summarize our findings and their implications obtained from the comparative analysis of statistical measures and their covariance and correlation matrices in BIGBOX and TILED simulations.

\begin{itemize}
    \item Comparing mean values of statistical measures between BIGBOX and TILED simulations revealed consistent results within $1\%$ across all statistics, redshifts, and multipoles. This agreement underscores the reliability of the simulations in capturing the underlying cosmological model.
    \item However, variance ratios showed $10\%$ to $30\%$ differences between the simulations, with the largest deviations occurring at high multipoles and redshifts. These discrepancies were attributed to super-sample covariance, which arises from large-scale modes present in BIGBOX but absent in TILED. 
    \item Shape noise was found to disturb statistical measures, particularly in the angular power spectrum and peak/minima counts, with deviations increasing at higher noise levels (e.g., DES and HSC). Nevertheless, super-sample covariance remained the dominant effect, maintaining its influence over noise contributions. Correlation matrices exhibited minimal structural changes ($<5\%$), demonstrating the robustness of the covariance framework against noise.
    \item Varying Gaussian smoothing scales primarily altered small-scale features in convergence maps without significantly impacting the covariance and correlation matrices. Higher-order statistics showed only modest changes. These results support the use of smoothing in weak lensing analyses to suppress small-scale noise without compromising the overall covariance structure.
    \item Box replication artifacts posed a systematic challenge, leading to the underestimation of mean angular power spectra and distortion of $\nu$-binned statistics in Replication-Influenced Patches (RIP). Covariance and correlation ratios in RIPs were consistently lower than in Replication-Minimal Patches (RMPs), indicating higher covariance in RIPs.
    Moreover, box replication artifacts introduce unique covariance structures in $10\%$ level against the BIGBOX, which is not present in RMPs of TILED.
    Additionally, box replication obscured redshift dependence, masking super-sample variance effects. These findings highlight the importance of addressing box replication artifacts to avoid biases in simulation-based analyses.
    \item Comparing covariance and correlation matrices from BIGBOX and TILED revealed that super-sample covariance significantly affects the variance of statistical measures, especially at high multipoles and redshifts. While mean values between the simulations were consistent within $1\%$, variance ratios showed notable differences due to the presence of large-scale modes in BIGBOX that are absent in TILED. This underscores the necessity of accounting for super-sample covariance in high-precision cosmological analyses.
\end{itemize}

Our results have significant implications for weak lensing studies in surveys such as DES, HSC, LSST, Euclid, and Roman. The prominence of super-sample covariance necessitates the inclusion of large-scale modes in simulations to ensure accurate variance modeling. The demonstrated robustness of covariance structures to shape noise and smoothing scales enhances confidence in their applicability across various observational conditions. However, mitigating systematic effects like box replication artifacts remains crucial to prevent biases in higher-order statistics interpretation.

Future research should focus on:
\begin{itemize}
    \item Developing methods to mitigate box replication artifacts, especially in simulations utilizing replicated volumes.
    \item Extending the analysis to incorporate additional cosmological models and baryonic effects to enhance simulation realism.
    \item Integrating these insights into survey design and data analysis pipelines to improve the precision of cosmological parameter inference.
\end{itemize}

By advancing simulation techniques and observational strategies, weak lensing continues to be a powerful tool for probing the Universe's fundamental properties and constraining cosmological models with unprecedented accuracy.