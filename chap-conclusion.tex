We conduct a high-resoluted large-scale simulations study on the effect og supe-sample effects on the covariance matrix of higher-order statistics. We compare two sets of simulations: BIGBOX simulation and TILED simulation. The BIGBOX simulations are simulations with large box size containing large-scale modes, while the TILED simulations are simulations with small box size replicated to cover the same volume of the BIGBOX simulations and without large-scale mode included. We compared the mean value, covariance matices and correlation matrices of higher-order statistics: bispectrum, PDF, peak/minima counts and Minkowski Functionals. Additionally, we measure those statistics with different noise levels and smoothing scales. We then compare the results from the BIGBOX simulations and the TILED simulations to see how the super-sample effect affects the covariance matrix of higher-order statistics. Finally, we discuss the implications of the super-sample effect on the covariance matrix of higher-order statistics and how to mitigate the effect in the future analysis.

The main results of this study are:
\begin{itemize}
    \item The super-sample effect affects the covariance matrix of higher-order statistics at the level of 10\% to 20\% in covariance matrix, and only few percent in correlation matrix. This means that the super-sample effect increase the overall correlation between different convergence values. Note that the box replication effect cause underestimation of the covariance matrix. Therefore, the value we found is the upper limit of the super-sample effect.
    \item The super-sample effect is universal for all higher-order statistics we tested and for all noise levels and smoothing scales. This means that the super-sample effect is not specific to any particular statistics or any particular noise level or smoothing scale. 
\end{itemize}

For future work, we suggest the following: