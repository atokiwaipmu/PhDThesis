This chapter compares a suite of weak lensing statistics derived from two sets of simulations: BIGBOX and TILED. Our goal is to quantify differences in mean values, covariance, and correlation matrices to assess how super-sample covariance and other systematic effects influence these measurements. We will also examine how shape noise, smoothing scales, and box replication artifacts affect the results.

\section{Comparison of Mean and Variance for Statistical Measures}
We compare the mean values and variances of various statistical measures derived from the BIGBOX and TILED simulations. The results show that the mean values are generally consistent between the two simulations, with differences below $1\%$ for most statistics. However, deviations are observed at low $\nu$ values for peak counts, minima, and Minkowski Functionals, indicating limitations in resolving low-density regions.

Variance comparisons reveal that the two simulations exhibit similar levels of statistical fluctuations, with no significant discrepancies. This suggests that the variance of the statistical measures is robust against differences in simulation setups.

\begin{figure}[ht]
    \centering
    \includegraphics[width=\textwidth]{figures/results/ell_main.png}
    \caption[Comparison of the mean and variance of $C^{\kappa\kappa}_{\ell}$ and Bispectrum]{Comparison of the mean and variance of the angular power spectrum ($C^{\kappa\kappa}_{\ell}$) and bispectrum of three configurations ($B_{\ell}^{(eq)}, B_{\ell}^{(iso)}, B_{\ell}^{(sq)}$) between the BIGBOX and TILED simulations for multiple source redshifts ($z_s = 0.5, 1.0, 1.5, 2.0, 2.5$).}
    \label{fig:ell_main}
\end{figure}

\begin{figure}[p]
    \centering
    \includegraphics[width=\textwidth]{figures/results/nu_main.png}
    \caption[Comparison of the mean and variance of Non-Correlation Statistics]{Comparison of the mean and variance of the probability density function (PDF), peak/minima counts, and Minkowski Functionals ($V_0, V_1, V_2$) between the BIGBOX and TILED simulations for multiple source redshifts ($z_s = 0.5, 1.0, 1.5, 2.0, 2.5$).}
\end{figure}

\newpage

\section{Detailed Analysis of Covariance and Correlation Matrices}
Examining covariance matrices reveals that BIGBOX tends to produce systematically higher covariance values than TILED, particularly at higher source redshifts. Correlation matrices highlight that, for some statistics, large-scale modes outside the sampled volume induce significant off-diagonal correlations. The angular power spectrum, in particular, shows increasing off-diagonal elements that align with theoretical expectations of super-sample effects.

\section{Effects of Noise on Statistical Measures}
Introducing various levels of shape noise shows that while certain statistics, such as $C_\ell^{\kappa\kappa}$ and Minima Counts, are sensitive to noise-induced changes, other non-Gaussian measures remain relatively stable. This indicates that some statistical observables are more robust against observational uncertainties and that careful noise modeling is essential when interpreting cosmological signals.

\section{Influence of Smoothing Scale}
We investigate how applying different Gaussian smoothing scales alters the statistical measures. As smoothing increases, small-scale features are diminished and the distribution of signal intensities changes. Covariance ratios become more unstable as structures blur, underscoring that the choice of smoothing scale affects both the amplitude and the covariance patterns of measured statistics.

\section{Identifying and Addressing Systematic Effects}
\subsection{Box Replication Effect}
We isolate patches most affected by the repeated tiling of the simulation box and find that these regions introduce biases into mean values and inflations in covariance. Removing or carefully handling these regions reduces discrepancies between BIGBOX and TILED, demonstrating the importance of mitigating this systematic.

\subsection{Flat-sky vs. Full-sky}
We test how well the flat-sky approximation holds by comparing statistics measured over small patches to their full-sky counterparts. While adequate for limited sky areas, the approximation breaks down at larger scales, affecting both mean values and covariance structures

\section{Summary}
Our findings confirm that super-sample covariance significantly impacts higher-order weak lensing statistics. The elevated covariance in BIGBOX simulations, sensitive dependence on source redshift, and systematic biases from box replication and smoothing all underscore the complexities in accurately modeling covariance. These insights are directly relevant for upcoming surveys, where understanding and accounting for such effects will be crucial for precise cosmological inference.



\section{Overview}
Figures~\ref{fig:cl_main} through~\ref{fig:mfs_cov} provide detailed visualizations of the mean values, variances, covariance matrices, and correlation matrices for each statistical measure under consideration. For each statistic, one figure illustrates the comparison of mean values and variances, while another figure presents the comparison of covariance and correlation matrices. 

From these figures, we observe that the mean values of most statistical measures exhibit excellent agreement between the BIGBOX and TILED simulations, with differences remaining below $1\%$ across the majority of the studied range. However, notable deviations occur at low $\nu$ values for peak counts, minima, and the Minkowski Functionals $V_1$ and $V_2$. These deviations are attributed to the limited resolution of the simulations, which affects the accurate detection of regions with the lowest density contrasts.

Analyzing the covariance matrices reveals that, except for the bispectrum, the ratios of covariance matrix elements between the BIGBOX and TILED simulations are consistently greater than unity. This indicates that the BIGBOX simulations yield higher covariance values compared to the TILED simulations, and this discrepancy becomes more pronounced at higher source redshifts. The bispectrum, on the other hand, exhibits noisy covariance matrices without a clear trend, making it challenging to draw definitive conclusions for this statistic.

Examining the correlation matrices further, we focus on the off-diagonal elements to assess the degree of inter-bin correlations. For statistical measures that are not inherently correlated, the off-diagonal elements remain close to unity, as expected. In contrast, the power spectrum shows off-diagonal elements that exceed unity, displaying a clear increasing trend with higher source redshifts. This behavior aligns with theoretical predictions of super-sample covariance effects, as detailed in \citet{PhysRevD.87.123504}, suggesting that larger-scale modes beyond the survey volume contribute to the observed correlations.

Overall, these findings support the hypothesis that super-sample covariance significantly impacts the statistical measures derived from our simulations. The discrepancies observed between the BIGBOX and TILED simulations emphasize the importance of considering super-sample effects in cosmological analyses. We will explore these effects in greater depth and seek further validation in the subsequent discussion chapter.

\begin{figure}[p]
    \centering
    \includegraphics[width=\textwidth]{figures/results/cl_main.png}
    \caption[$C_\ell^{\kappa\kappa}$ Mean and Variance]{Comparison of the mean values of the angular power spectrum ($C^{\kappa\kappa}_{\ell}$) for different source redshifts ($z_s = 0.5, 1.0, 1.5, 2.0, 2.5$) obtained from the BIGBOX (solid lines) and TILED (dashed lines) simulations. The lower subplots show the ratio of the TILED to BIGBOX mean values, with a reference line at unity to facilitate the assessment of agreement between the two simulations.}
    \label{fig:cl_main}
    \vspace{2cm}
    \includegraphics[width=\textwidth]{figures/results/cl_cov.png}
    \caption[$C_\ell^{\kappa\kappa}$ Correlation and Covariance Ratio]{Comparison of the covariance matrices and correlation matrices of the angular power spectrum ($C^{\kappa\kappa}_{\ell}$) between the BIGBOX and TILED simulations for various source redshifts ($z_s = 0.5, 1.0, 1.5, 2.0, 2.5$). The displayed ratios represent the element-wise division of the covariance and correlation matrices from the TILED simulations by those from the BIGBOX simulations. The `avg' denotes the average ratio of the considered matrix elements.}
    \label{fig:cl_cov}
\end{figure}

\begin{figure}[p]
    \centering
    \includegraphics[width=\textwidth]{figures/results/bl_main.png}
    \caption[Bispectrum Mean and Variance]{Same as Figure~\ref{fig:cl_main}, but for the bispectrum. }
    \label{fig:bl_main}
\end{figure}

\begin{figure}[p]
    \centering
    \includegraphics[width=\textwidth]{figures/results/bl_corr.png}
    \caption[Bispectrum Correlation and Covariance]{Similar to Figure~\ref{fig:cl_cov}, but for the correlation (upper) and covariance (lower) of bispectrum. The noisy nature of the bispectrum covariance makes it challenging to discern clear trends between the simulations.}
    \label{fig:bl_cov}
    \vspace{0.5cm}
    \includegraphics[width=\textwidth]{figures/results/bl_cov.png}
\end{figure}

\begin{figure}[p]
    \centering
    \includegraphics[width=\textwidth]{figures/results/pdf_main.png}
    \caption[PDF Mean and Variance]{Same as Figure~\ref{fig:cl_main}, but for the probability density function (PDF) of the convergence field. The comparison highlights the agreement in mean PDF values between the simulations across different redshifts.}
    \label{fig:pdf_main}
    \vspace{2cm}
    \includegraphics[width=\textwidth]{figures/results/pdf_cov.png}
    \caption[PDF Covariance]{Same as Figure~\ref{fig:cl_cov}, but for the covariance matrices of the PDF. The covariance ratios indicate higher covariance in the BIGBOX simulations, particularly at higher redshifts.}
    \label{fig:pdf_cov}
\end{figure}

\begin{figure}[p]
    \centering
    \includegraphics[width=\textwidth]{figures/results/peaks_main.png}
    \caption[Peak Counts Mean and Variance]{Same as Figure~\ref{fig:cl_main}, but for peak counts in the convergence maps. The analysis reveals deviations at low $\nu$ values due to resolution limitations affecting low-density regions.}
    \label{fig:peak_main}
    \vspace{2cm}
    \includegraphics[width=\textwidth]{figures/results/peaks_cov.png}
    \caption[Peak Counts Covariance]{Same as Figure~\ref{fig:cl_cov}, but for the covariance matrices of peak counts. The covariance ratios suggest increased covariance in the BIGBOX simulations, with pronounced effects at higher redshifts.}
    \label{fig:peak_cov}
\end{figure}

\begin{figure}[p]
    \centering
    \includegraphics[width=\textwidth]{figures/results/minima_main.png}
    \caption[Minima Counts Mean and Variance]{Same as Figure~\ref{fig:cl_main}, but for minima in the convergence maps. The comparison underscores the simulation's limitations at resolving low-density minima accurately.}
    \label{fig:min_main}
    \vspace{2cm}
    \includegraphics[width=\textwidth]{figures/results/minima_cov.png}
    \caption[Minima Counts Covariance]{Same as Figure~\ref{fig:cl_cov}, but for the covariance matrices of minima. The covariance ratios reflect higher values in the BIGBOX simulations, consistent with other statistical measures.}
    \label{fig:min_cov}
\end{figure}

\begin{figure}[p]
    \centering
    \includegraphics[width=\textwidth]{figures/results/mfs_main.png}
    \caption[Minkowski Functionals Mean and Variance]{Same as Figure~\ref{fig:cl_main}, but for Minkowski Functionals (area $V_0$, perimeter $V_1$, and genus $V_2$). The agreement in mean values between simulations is generally good, with some discrepancies at extreme density thresholds.}
    \label{fig:mfs_main}
\end{figure}

\begin{figure}[p]
    \centering
    \includegraphics[width=\textwidth]{figures/results/mfs_corr.png}
    \caption[Minkowski Functionals Correlation and Covariance]{Similar to Figure~\ref{fig:cl_cov}, but for the correlation (upper) and covariance (lower) of Minkowski Functionals.}
    \label{fig:mfs_cov}
    \vspace{0.5cm}
    \includegraphics[width=\textwidth]{figures/results/mfs_cov.png}
\end{figure}

\section{Effects of Noise}
To assess the impact of observational noise, we have introduced five different shape noise levels into the simulations. Due to the significant influence of noise on higher-order statistics, the bispectrum has been excluded from this part of the analysis.

Figures~\ref{fig:avg_noise_cov} and~\ref{fig:avg_noise_corr} illustrate how the average ratios of covariance matrices and correlation matrices change with varying shape noise levels. Except for the angular power spectrum, the non-Correlation statistics exhibit stable covariance ratios across different noise levels. 

\begin{figure}[p]
    \centering
    \includegraphics[width=\textwidth]{figures/results/avg_cov_ratio_ngal.png}
    \caption[Average BIGBOX/TILED Ratio of Covariance for multiple noise levels]{Average ratio of covariance matrices of statistical measures between the BIGBOX and TILED simulations for different shape noise levels (see Table~\ref{tab:survey_comparison}). The increasing trend indicates does not affected by the noise level.}
    \label{fig:avg_noise_cov}
    \vspace{2cm}
    \includegraphics[width=\textwidth]{figures/results/avg_corr_ratio_ngal.png}
    \caption[Average BIGBOX/TILED Ratio of Correlation for multiple noise levels]{Same as Figure~\ref{fig:avg_noise_cov}, but for the correlation matrices. The off-diagonal elements compared to the diagonal elements do not show a clear trend with noise levels.}
    \label{fig:avg_noise_corr}
\end{figure}

\begin{figure}[p]
    \centering
    \includegraphics[width=\textwidth]{figures/results/avg_cov_ratio_sl.png}
    \caption[Average BIGBOX/TILED Ratio of Covariance for multiple smoothing scales]{Average ratio of covariance matrices of statistical measures between the BIGBOX and TILED simulations for different smoothing scales. Larger smoothing scales lead to increased discrepancies in covariance estimates due to the loss of small-scale information.}
    \label{fig:avg_sl_cov}
    \vspace{2cm}
    \includegraphics[width=\textwidth]{figures/results/avg_corr_ratio_sl.png}
    \caption[Average BIGBOX/TILED Ratio of Correlation for multiple smoothing scales]{Same as Figure~\ref{fig:avg_sl_cov}, but for the correlation matrices. The instability at larger smoothing scales reflects the challenges in capturing correlations at reduced resolutions.}
    \label{fig:avg_sl_corr}
\end{figure}

Figures~\ref{fig:cl_noise} and~\ref{fig:ng_noise} demonstrate how the ratios of covariance matrices for the angular power spectrum and the non-correlation statistics change with different shape noise levels. The results indicate that the angular power spectrum and minima are particularly sensitive to the shape noise level, exhibiting significant variations in their covariance matrices. In contrast, other non-correlation statistics remain more robust against changes in the shape noise level, maintaining relatively stable off-diagonal elements in their covariance matrices.

\begin{figure}[p]
    \centering
    \includegraphics[width=0.6\textwidth]{figures/results/correlation_cov_noise.png}
    \caption[BIGBOX/TILED Ratio of Covariance Matricies for multiple noise levels: $C^{\kappa\kappa}_{\ell}$]{Ratio of covariance matrices of the angular power spectrum ($C^{\kappa\kappa}_{\ell}$) between the BIGBOX and TILED simulations for different shape noise levels (see Table~\ref{tab:survey_comparison}). The sensitivity of the power spectrum to noise is evident from the fluctuating covariance ratios with higher noise levels.}
    \label{fig:cl_noise}
    \vspace{0.5cm}
    \includegraphics[width=0.8\textwidth]{figures/results/nongaussian_cov_noise.png}
    \caption[BIGBOX/TILED Ratio of Covariance Matricies for multiple noise levels: Non-Gaussian Statistics]{Same as Figure~\ref{fig:cl_noise}, but for the non-Gaussian statistical measures. The robustness of these measures against noise variations is reflected in the relatively stable covariance ratios.}
    \label{fig:ng_noise}
\end{figure}

\section{Effects of Smoothing Scale}
To evaluate the impact of smoothing on the statistical measures, we have applied four different smoothing scales to the simulations. Smoothing affects the resolution of the convergence maps and can influence the detection of small-scale structures.

Figures~\ref{fig:avg_sl_cov} and~\ref{fig:avg_sl_corr} show how the average ratios of covariance matrices and correlation matrices change with varying smoothing scales. The ratios become more unstable due to the smoothing effect washing out small-scale structures.

Figure~\ref{fig:ng_smoothing} illustrates the effects of smoothing scale on non-Correlation statistical measures. As the smoothing scale increases, the finer structures in the convergence maps are blurred, leading to changes in the statistical properties. The blank bins that previously contained little or no signal begin to be filled due to the spread of signals from neighboring bins, while the overall signal intensity is redistributed.

\begin{figure}[ht]
    \centering
    \includegraphics[width=0.8\textwidth]{figures/results/nongaussian_cov_sl.png}
    \caption[BIGBOX/TILED Ratio of Covariance Matricies for multiple smoothing scales: Non-Gaussian Statistics]{Same as Figure~\ref{fig:ng_noise}, but showing the impact of different smoothing scales on the covariance matrices of non-Gaussian statistical measures. The results emphasize how increased smoothing affects the detection and characterization of small-scale features.}
    \label{fig:ng_smoothing}
\end{figure}

\section{Source of Systematics}
In this section, we will discuss the possible sources of systematics that could affect the covariance matrix. 

\subsection{Box Replication Effect}\label{sec:boxreplication}
The box replication effect arises from replication of the simulation box in order to increase the volume of the simulation. In our setup, patches around special directions suffer the most from the box replication effect. To roughly assess the effect, we can compare the covariance matrices made from the patches of those directions and the rest of the patches.

Here, we defined the directions where are suspected to be affected by the box replication effect by the following criteria:
\begin{itemize}
    \item data points around equator: $ \left| \theta_i - \frac{\pi}{2} \right| \leq R_{\text{patch}} $
    \item data points around the edges of octant: $ \left| \phi_i - \frac{k\pi}{2} \right| \leq R_{\text{patch}} $ for $k=0,1,2,3$
\end{itemize}
where $(\theta_i, \phi_i)$ denotes the center of the patch $i$, and $R_{\text{patch}} = 5\sqrt{2}\, \mathrm{\deg}$ is the half diagonal length of the patch. 

We found that the patches including the point $(\theta_i, \phi_i) = (\pi/2, 0)$ have a signnificantly variated mean and variance, even comapared to the rest of suspects. Therefore, we simply discarded this point as it will bias the result. 

Finally, we obtain $70$ patches from each realization, and $1400$ patches for TILED simulations and $770$ patches for BIGBOX simulations in total.

Figure~\ref{fig:boxreplication_cov} and Figure~\ref{fig:boxreplication_corr} show the average ratios of the covariance matrices and correlation matrices between BIGBOX and TILED simulations. Regardless the noisy behavior of bispectrum, the ratio are much closer to unity for the suspected patches. This implies that the box replication effect is significantly enhance the covariance matrix.

\begin{figure}[ht]
    \centering
    \includegraphics[width=\textwidth]{figures/BR_ratio_cov.png}
    \caption{The BIGBOX / TILED ratios of covariance matrices, for the case of the patches around special directions and the rest of the patches. }
    \label{fig:boxreplication_cov}
    \includegraphics[width=\textwidth]{figures/BR_ratio_corr.png}
    \caption{The same as Figure~\ref{fig:boxreplication_cov}, but for correlation matrices.}
    \label{fig:boxreplication_corr}
\end{figure}

Figure~\ref{fig:boxreplication_ell} and~\ref{fig:boxreplication_nu} show the comparison of the ratios of mean and variance of each Statistics. Clearly, the suspected patches have biased mean values. The variance ratios are closer to unity, which means that the variance of the suspected patches is larger than the rest of the patches.

\begin{figure}[p]
    \centering
    \includegraphics[width=0.9\textwidth]{figures/BR/BR_ratio_ell.png}
    \caption{The ratios of mean and variance of power spectrum and bispectrum between the patches around special directions and the rest of the patches. The mean of suspected patches are biased and the variance is larger.}
    \label{fig:boxreplication_ell}
    \includegraphics[width=0.9\textwidth]{figures/BR/BR_ratio_nu.png}
    \caption{The same as Figure~\ref{fig:boxreplication_ell}, but for PDF, peak/minima counts and Minkowski functionals. The suspected patches tend to have more extreme values and larger variance.}
    \label{fig:boxreplication_nu}
\end{figure}

\section{flat-sky vs. full-sky}
The flat-sky approximation is a good approximation for small patches on the sky. However, the flat-sky approximation breaks down for large patches. In this section, we conduct a test for angular power spectrum, PDF, and peak/minima counts to see how the flat-sky approximation affects the statistics.