\section{Nancy Grace Roman Space Telescope}

\subsection{Overview and Mission Objectives}

The \emph{Nancy Grace Roman Space Telescope} (hereafter referred to as \emph{Roman}; formerly designated as the \emph{Wide-Field Infrared Survey Telescope}, \textit{WFIRST}; \citealt{2015arXiv150303757S}) represents a seminal endeavor within NASA's astrophysics portfolio. This mission leverages a 2.4-meter aperture telescope, originally developed for national reconnaissance purposes, repurposing its advanced instrumentation and observational capabilities toward forefront astronomical research. The \emph{Roman} mission is designed to address key questions in cosmology and exoplanet science, thereby significantly enhancing our understanding of the Universe's expansion history, the distribution of dark matter and dark energy, and the prevalence of planetary systems in our galaxy.

\subsection{Surveys and Scientific Goals}

The \emph{Roman} Space Telescope's scientific program is anchored by two primary surveys: the \emph{High Latitude Survey} and the \emph{Galactic Bulge Time Domain Survey}, each meticulously designed to achieve specific scientific objectives.

\subsubsection{High Latitude Survey}

A cornerstone of the \emph{Roman}'s scientific agenda is the \textbf{High Latitude Survey}, engineered to probe the enigmatic phenomenon of dark energy, which is hypothesized to drive the accelerated expansion of the Universe~\cite{1998AJ....116.1009R, 1999ApJ...517..565P}. This survey integrates a tripartite observational strategy to examine both the temporal evolution of cosmic expansion and the growth of large-scale structures. The employed methodologies encompass:

\begin{itemize}
    \item \textbf{Supernova Surveys}: Utilizing Type Ia supernovae as standard candles to trace the expansion history of the Universe~\cite{2011ApJ...737..102S}.
    \item \textbf{Weak Gravitational Lensing Imaging}: Mapping the distribution of dark matter through the subtle distortions of background galaxies~\cite{2001PhR...340..291B}.
    \item \textbf{Spectroscopic Surveys of Baryon Acoustic Oscillations}: Measuring the scale of baryon acoustic oscillations to serve as a cosmological standard ruler~\cite{2005ApJ...633..560E}.
\end{itemize}

These techniques collectively furnish a comprehensive understanding of the Universe's expansion history, thereby elucidating the properties and dynamics of dark matter and dark energy. The High Latitude Survey is anticipated to cover thousands of square degrees, providing high-resolution imaging and spectroscopy in the near-infrared wavelengths.

\subsubsection{Galactic Bulge Time Domain Survey}

Complementing the High Latitude Survey, the \textbf{Galactic Bulge Time Domain Survey} employs the microlensing technique to detect and characterize a substantial population of exoplanets, including those with masses as low as that of Mars~\cite{2019ApJS..241....3P}. Focusing on stellar populations within the Milky Way's bulge, this survey enhances our comprehension of planetary formation and distribution mechanisms in environments with high stellar density. By monitoring millions of stars over time, the \emph{Roman} Space Telescope aims to detect the gravitational microlensing events caused by planets orbiting both luminous and dark host stars, thus expanding the demographic census of exoplanets~\cite{2012ARA&A..50..411G}.

\subsection{Instrumentation and Capabilities}

The \emph{Roman} Space Telescope is equipped with a 2.4-meter primary mirror, comparable in size to the \emph{Hubble Space Telescope}, but with a significantly larger field of view—approximately 100 times greater~\cite{2015arXiv150303757S}. The primary instruments include:

\begin{itemize}
    \item \textbf{Wide Field Instrument (WFI)}: A near-infrared imager and spectrometer that provides high-resolution imaging and integral-field spectroscopy over a large field of view.
    \item \textbf{Coronagraph Instrument}: A technology demonstration instrument designed to perform high-contrast imaging and spectroscopy of exoplanets and circumstellar disks.
\end{itemize}

These advanced instruments enable the \emph{Roman} Space Telescope to perform wide-field surveys with unprecedented sensitivity and resolution in the near-infrared regime.

\subsection{Expected Scientific Contributions}

The \emph{Roman} Space Telescope is poised to make transformative contributions to astrophysics and cosmology. In addition to its primary surveys, it will offer opportunities for a \emph{Guest Observer} program, allowing the broader scientific community to propose observations for a wide range of astrophysical phenomena~\cite{2019arXiv190205569A}. The mission is expected to:

\begin{itemize}
    \item Provide precise measurements of cosmological parameters, refining our understanding of dark energy and the geometry of the Universe~\cite{2018ApJ...867...23H}.
    \item Expand the census of exoplanets, particularly those in the outer regions of planetary systems, thereby informing models of planetary formation and evolution~\cite{2019ApJS..241....3P}.
    \item Advance the study of galaxy formation and evolution through deep imaging and spectroscopy of distant galaxies~\cite{2018MNRAS.477.5382B}.
    \item Enhance our understanding of the Milky Way's structure and stellar populations~\cite{2020AJ....160..123J}.
\end{itemize}

By addressing these fundamental questions, the \emph{Roman} Space Telescope will significantly augment our knowledge of the cosmos and lay the groundwork for future astronomical discoveries.
