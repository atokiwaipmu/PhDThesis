\section{Subaru Hyper Suprime-Cam}

\subsection{Instrument Overview}

The \emph{Subaru Hyper Suprime-Cam} (HSC; \citealt{2018PASJ...70S...1M}) is a state-of-the-art wide-field imaging instrument mounted at the prime focus of the 8.2-meter Subaru Telescope located on Maunakea, Hawaii. The HSC features an expansive $1.5^\circ$ diameter field of view, achieved through an array of 104 charge-coupled devices (CCDs), each with $2048 \times 4096$ pixels, culminating in a total of approximately 870 megapixels. This sophisticated configuration enables the acquisition of high-fidelity, wide-field optical data, augmented by a precision corrector lens system that preserves image quality across the entire observational field. The substantial field of view, combined with Subaru's significant light-collecting capabilities, renders the HSC exceptionally well-suited for deep and wide imaging surveys that are critical for studies of \emph{weak gravitational lensing} and the \emph{large-scale structure} of the Universe.

The instrument is equipped with five broad-band filters—\textit{g}, \textit{r}, \textit{i}, \textit{z}, and \textit{y} (\citealt{2018PASJ...70...66K})—alongside several narrow-band filters that cover optical to near-infrared wavelengths. Benefiting from the exceptional atmospheric conditions at Maunakea and the high-resolution optics of the Subaru Telescope, the HSC achieves a median seeing of $0.6$ arcseconds in the $i$-band, facilitating detailed and precise astronomical analyses.

\subsection{Survey Design: The Hyper Suprime-Cam Subaru Strategic Program}

The \emph{Hyper Suprime-Cam Subaru Strategic Program} (HSC-SSP; \citealt{2018PASJ...70S...4A}) is a comprehensive, multi-band imaging survey that exploits the advanced capabilities of the HSC instrument. Initially allocated 300 observing nights over a five-year period starting in March 2014, the program's duration was extended to 340 nights to compensate for observational losses due to weather conditions, technical issues, and seismic activity. The HSC-SSP is meticulously designed to address a broad spectrum of astrophysical inquiries, including but not limited to cosmology, galaxy evolution, and solar system science.

The survey is stratified into three distinct layers, each engineered to achieve specific scientific objectives through varying degrees of sky coverage and imaging depth:

\begin{itemize}
    \item \textbf{Wide Layer}: Encompassing approximately 1,400 square degrees around the celestial equator, the Wide layer attains a depth of $i \sim 26$ magnitudes. This extensive coverage is optimized for mapping large-scale structures and conducting statistical studies of weak gravitational lensing, thereby contributing significantly to our understanding of dark matter distribution and cosmic acceleration.

    \item \textbf{Deep Layer}: Covering roughly 28 square degrees across four distinct fields, the Deep layer reaches a depth of $i \sim 27$ magnitudes. This intermediate configuration balances area and depth, supporting research on galaxy evolution, supernova detection, and the statistical properties of faint astronomical objects.

    \item \textbf{UltraDeep Layer}: Targeting two fields that collectively cover 3.5 square degrees, the UltraDeep layer achieves an impressive depth of $i \sim 28$ magnitudes. Utilizing both broad-band and narrow-band filters, this layer is dedicated to probing the high-redshift Universe, facilitating investigations into early galaxy formation, quasars, and the epoch of reionization, thereby providing critical insights into the early phases of cosmic evolution.
\end{itemize}

\subsection{Data Releases and Processing}

To date, the HSC-SSP has disseminated several public data releases—DR1, DR2, and DR3 (\citealt{2018PASJ...70S...8A, 2019PASJ...71..114A, 2022PASJ...74..247A})—each incorporating progressively larger datasets, enhanced calibration procedures, and refined data processing methodologies. The data processing pipeline is managed using the \emph{Legacy Survey of Space and Time} (LSST) Science Pipelines, which provide high-quality analysis tools accessible to the astronomical community. These tools generate calibrated imaging data, comprehensive photometric catalogs, and advanced data products such as photometric redshifts and precise shape measurements. Consequently, the HSC-SSP data releases empower a wide array of scientific investigations and promote collaborative advancements within the field of astronomy.
