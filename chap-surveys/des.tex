\section{Dark Energy Survey}

\subsection{Overview and Objectives}

The \emph{Dark Energy Survey} (DES; \citealt{2005astro.ph.10346T, 2018ApJS..239...18A, 2021ApJS..255...20A}) was meticulously designed as a comprehensive, wide-field imaging survey operating in optical and near-infrared wavelengths, with the primary objective of probing the nature of cosmic acceleration and the underlying mechanism of \emph{dark energy}. Spanning approximately $5{,}000$ square degrees of the Southern Galactic Cap, DES was conducted over a six-year period from August 2013 to January 2019. The survey utilized the $570$-megapixel \emph{Dark Energy Camera} (DECam; \citealt{2015AJ....150..150F}), which is mounted on the 4-meter \emph{Blanco Telescope} at the Cerro Tololo Inter-American Observatory (CTIO) in Chile.

\subsection{Survey Components}

DES observations were strategically divided into two primary components: the \emph{Wide-Area Survey} and the \emph{Supernova Survey}, each meticulously tailored to address specific scientific goals within the overarching framework of DES.

\subsubsection{Wide-Area Survey}

The Wide-Area Survey constituted the principal observational effort of DES, covering approximately $5{,}000$ square degrees across five photometric bands ($g$, $r$, $i$, $z$, $Y$). This extensive footprint was purposefully selected to overlap with other significant surveys, such as the \emph{South Pole Telescope Survey} and the \emph{Sloan Digital Sky Survey} (SDSS) Stripe 82~\cite{2009ApJS..182..543A}, thereby enhancing calibration accuracy and enabling synergistic cross-survey analyses. The Wide-Area Survey aimed to map the large-scale structure of the Universe with high precision, facilitating studies on galaxy clustering, weak gravitational lensing, and the abundance of galaxy clusters, all of which are sensitive probes of dark energy and cosmological parameters.

\subsubsection{Supernova Survey}

Complementing the Wide-Area Survey, the Supernova Survey focused on a more concentrated region of approximately $27$ square degrees. This component employed a higher observational cadence, with images acquired in the $g$, $r$, $i$, and $z$ bands roughly every seven days. Such a cadence was specifically designed to capture transient phenomena, notably \emph{Type Ia supernovae}~\cite{2019PhRvL.122q1301A}, which serve as standard candles for cosmic distance measurements. By monitoring these supernovae over time, the Supernova Survey aimed to construct a Hubble diagram extending to high redshifts, thereby providing stringent constraints on the equation of state of dark energy.

\subsection{Instrumentation and Data Processing}

DECam, the instrument central to DES, features a focal plane array of 62 science CCDs, each with $2048 \times 4096$ pixels, providing a large field of view of $2.2$ degrees in diameter. The camera's design includes specialized optics to achieve uniform image quality across the field, essential for weak lensing measurements that require precise shape determinations of distant galaxies.

Data processing for DES was handled by the \emph{DES Data Management} (DESDM) system~\cite{2008SPIE.7016E..0LM}, which performed tasks ranging from image detrending and astrometric calibration to co-addition and catalog generation. Advanced algorithms were employed for photometric calibration, object detection, and classification, ensuring high-quality data products suitable for cosmological analyses.

\subsection{Scientific Contributions}

DES has made significant contributions to cosmology and astrophysics, including precise measurements of cosmic shear~\cite{2022PhRvD.105b3514A} and galaxy clustering~\cite{2022PhRvD.105b3520A}. The combination of these probes has led to improved constraints on cosmological parameters, particularly the matter density $\Omega_m$ and the amplitude of matter fluctuations $\sigma_8$. Furthermore, DES data have been instrumental in addressing tensions in cosmological measurements, such as the $S_8$ discrepancy, and in exploring possible extensions to the $\Lambda$CDM model.

