\section{Introduction to Astronomical Surveys}

\subsection{Purpose and Scientific Objectives}
Contemporary astronomical surveys constitute extensive and systematic observational endeavors designed to map and catalog vast regions of the sky with unparalleled depth and precision. These surveys yield comprehensive datasets that are crucial for addressing fundamental questions in astrophysics and cosmology. They facilitate investigations into the elusive nature of \emph{dark energy}, provide rigorous tests of the \emph{standard cosmological model}—the $\Lambda$CDM paradigm—and enable detailed studies of the formation and evolution of \emph{cosmic structures}. By delivering high-fidelity observational data, these surveys significantly advance our understanding of the Universe and the fundamental physical laws that govern it.

\paragraph{Testing the Standard Cosmological Model}
One of the primary objectives of astronomical surveys is to rigorously test the standard cosmological model, known as the $\Lambda$CDM model (Lambda Cold Dark Matter). While the $\Lambda$CDM model successfully accounts for a wide range of cosmological observations~\cite{Planck2018}, it encounters notable challenges such as the \emph{Hubble tension}—a significant discrepancy between measurements of the Universe's expansion rate derived from early-universe observations~\cite{Planck2018} and those obtained from late-universe observations~\cite{Riess2019}—and inconsistencies in parameters like the amplitude of matter clustering, denoted by $S_8$~\cite{Heymans2021}. These surveys endeavor to produce precise measurements of cosmological parameters, aiming to either corroborate the $\Lambda$CDM model or uncover deviations that could indicate new physics beyond the current paradigm. By juxtaposing high-precision observational data with theoretical predictions, astronomical surveys enhance our understanding of the fundamental forces and constituents that shape the cosmos.

\paragraph{Studying the Growth and Evolution of Cosmic Structures}
Another fundamental objective of astronomical surveys is to investigate the formation and evolution of cosmic structures, including galaxies, galaxy clusters, and dark matter halos. By mapping the spatial distribution of millions of galaxies, these surveys construct detailed maps of dark matter over vast cosmological volumes. Techniques such as \emph{weak gravitational lensing}—which measures subtle distortions in the shapes of background galaxies caused by the gravitational potential of intervening mass distributions—illuminate the role of dark matter in the process of structure formation~\cite{Bartelmann2001}. Moreover, \emph{galaxy-galaxy lensing}, which correlates the lensing signal with the positions of foreground galaxies, provides insights into the connection between luminous matter and the underlying dark matter distribution~\cite{Mandelbaum2017}. These comprehensive maps enable stringent tests of theoretical predictions from various cosmological models regarding the large-scale structure of the Universe.

\subsection{Imaging vs.\ Spectroscopic Surveys}
Astronomical surveys can be broadly categorized into \emph{imaging surveys} and \emph{spectroscopic surveys}, each employing distinct methodologies to capture and analyze celestial phenomena.

\subsubsection{Imaging Surveys}
Imaging surveys acquire wide-field images of the sky across multiple wavelengths, providing spatial data that enable astronomers to map cosmic structures, identify transient events, and analyze galaxy populations. By systematically scanning large regions of the sky, these surveys generate extensive catalogs documenting object positions, apparent magnitudes, and morphological characteristics. The varying depths and angular resolutions of these surveys facilitate the detection of both nearby and distant astronomical objects, thereby supporting detailed morphological and statistical studies across cosmic time. Prominent examples of imaging surveys include the \textit{Sloan Digital Sky Survey} (SDSS; \citealt{2019BAAS...51g.274K}), the \textit{Dark Energy Survey} (DES; \citealt{2018ApJS..239...18A}), and the forthcoming \textit{Legacy Survey of Space and Time} (LSST; \citealt{2019ApJ...873..111I}).

\subsubsection{Spectroscopic Surveys}

Spectroscopic surveys collect detailed spectral data from individual celestial objects by dispersing their emitted light into constituent wavelengths, thereby revealing redshifts, chemical compositions, and kinematic properties. These measurements are indispensable for studying galaxy dynamics, the distribution of dark matter, and the expansion history of the Universe. Unlike imaging surveys that capture broad swaths of the sky, spectroscopic surveys often target specific objects identified in imaging data, focusing on obtaining high-resolution spectral information. The spectral resolution and wavelength coverage are critical parameters that determine the precision of measurements such as redshift determinations and elemental abundances. Key spectroscopic surveys include the \textit{Baryon Oscillation Spectroscopic Survey} (BOSS; \citealt{2013AJ....145...10D}), which investigates baryon acoustic oscillations, the \textit{Dark Energy Spectroscopic Instrument} (DESI; \citealt{2016arXiv161100036D}), designed to map large-scale cosmic structure, and the \textit{Kilo-Degree Survey} (KiDS; \citealt{2013Msngr.154...44D}) with its spectroscopic extensions that enhance research on dark matter and dark energy.

\subsection{Ground-Based vs.\ Space-Based Surveys}

The operational platform of an astronomical survey—whether ground-based or space-based—fundamentally determines its observational capabilities, including wavelength coverage, angular resolution, and sensitivity.

\subsubsection{Ground-Based Surveys}
Ground-based surveys utilize telescopes located on Earth's surface, capitalizing on accessible infrastructure and enabling routine maintenance and upgrades of large, high-power instruments. These observatories can host a diverse array of instruments designed for observations across multiple wavelengths and facilitate prompt follow-up observations. However, atmospheric effects such as absorption, scattering, and turbulence degrade image quality and restrict observations at certain wavelengths, particularly in the ultraviolet and infrared regions. Notable ground-based surveys include the \emph{Hyper Suprime-Cam} (HSC; \citealt{2018PASJ...70S...4A}), the \emph{Dark Energy Survey} (DES; \citealt{2018ApJS..239...18A}), and the \emph{Kilo-Degree Survey} (KiDS; \citealt{2013Msngr.154...44D}).

\subsubsection{Space-Based Surveys}
Space-based surveys operate from satellites or space telescopes positioned above Earth's atmosphere, offering exceptional clarity and sensitivity across a wide range of wavelengths, especially in the ultraviolet and infrared bands that are largely inaccessible from the ground. Free from atmospheric distortions, these instruments achieve high angular resolution and can detect faint, distant objects through long-duration, stable observations. Despite these advantages, space missions are costly, require extensive international collaboration, and present limited opportunities for maintenance, upgrades, or repairs. Prominent examples include the \emph{Hubble Space Telescope} (HST; \citealt{2001ApJ...553...47F}), the forthcoming \emph{Nancy Grace Roman Space Telescope} (\emph{Roman}; \citealt{2015arXiv150303757S}), and the \emph{Gaia} mission~\citep{2016A&A...595A...2G}.

\subsection{Stage-III vs.\ Stage-IV Surveys}
\label{sec:survey-stages}

The classification of astronomical surveys into \emph{Stage-III} and \emph{Stage-IV} categories serves to distinguish successive generations based on technological sophistication, scale, and scientific objectives. While this framework is particularly pertinent in dark energy research~\cite{2006astro.ph..9591A}, it is broadly applicable across cosmology and astrophysics.

\subsubsection{Stage-III Surveys}
Stage-III surveys represent the current generation of large-scale astronomical projects, integrating advanced yet moderately scaled instrumentation and methodologies. Their primary aims include refining cosmological parameters, mapping large-scale structures, and deepening our understanding of dark energy and dark matter. Utilizing multi-band imaging and spectroscopy, Stage-III surveys achieve moderate sky coverage and depth, supported by sophisticated data processing pipelines capable of handling substantial datasets. Notable examples of Stage-III surveys include the \emph{Dark Energy Survey} (DES; \citealt{2018ApJS..239...18A}), the \emph{Kilo-Degree Survey} (KiDS; \citealt{2013Msngr.154...44D}), and the \emph{Hyper Suprime-Cam} survey (HSC; \citealt{2018PASJ...70S...4A}), all of which have provided high-quality data instrumental in advancing cosmological models.

\subsubsection{Stage-IV Surveys}
Stage-IV surveys represent the forthcoming generation of astronomical initiatives, characterized by unparalleled scale, depth, and precision. Leveraging cutting-edge technologies, these surveys aim to achieve high-precision cosmological measurements, explore dark energy and dark matter with greater fidelity, and uncover new astrophysical phenomena. They are distinguished by ultra-wide sky coverage, deep imaging and spectroscopy, high angular resolution, and the integration of multi-wavelength data. Advanced data management techniques are employed to process petabyte-scale datasets efficiently. Prominent examples include the \emph{Legacy Survey of Space and Time} (LSST) conducted by the Vera C.\ Rubin Observatory~\cite{2019ApJ...873..111I}, the \emph{Dark Energy Spectroscopic Instrument} (DESI; \citealt{2016arXiv161100036D}), and the forthcoming \emph{Nancy Grace Roman Space Telescope} (\emph{Roman}; \citealt{2015arXiv150303757S}), all of which are poised to transform our understanding of the cosmos and address critical questions in astrophysics.
