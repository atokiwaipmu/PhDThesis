\section{Legacy Survey of Space and Time}

\subsection{Overview and Scientific Objectives}

The \emph{Legacy Survey of Space and Time} (LSST; \citealt{2009arXiv0912.0201L, 2019ApJ...873..111I}), conducted by the Vera C.\ Rubin Observatory, is a groundbreaking optical survey meticulously designed to perform a comprehensive imaging campaign covering approximately half of the celestial sphere. Operating with six photometric filters (\emph{ugrizy}), LSST will systematically re-image its surveyed regions at intervals of a few nights over a planned ten-year observational period. This extensive survey is anticipated to produce an unparalleled dataset, comprising approximately $3.2 \times 10^{13}$ individual observations of around $2 \times 10^{10}$ galaxies, along with a comparable number of stellar objects. This vast repository of observations will enable a broad spectrum of scientific investigations, including studies in cosmology, Galactic structure, and time-domain astrophysics.

\subsection{Instrumentation and Survey Strategy}

To achieve its ambitious scientific objectives, LSST has been constructed as an advanced, wide-field observatory centered around an 8.4-meter primary mirror and a field of view spanning 9.6 square degrees. The system's observational capabilities are enhanced by a 3.2-gigapixel camera, facilitating high-throughput data acquisition. The unique optical design employs a three-mirror system to deliver a wide field of view with excellent image quality across the entire focal plane.

The survey primarily employs a \emph{deep-wide-fast} observing strategy, allocating approximately 90\% of the telescope's operational time to this primary mode~\cite{2009arXiv0912.0201L}. This approach aims to maximize both spatial coverage and temporal cadence across the main survey region, enabling the detection of transient phenomena and the mapping of large-scale structures. The remaining 10\% of LSST's observing time is dedicated to specialized programs designed to address specific astrophysical phenomena. These programs include \emph{deep drilling fields}, which involve intensified observations in selected regions to achieve greater depth, and targeted surveys of dynamically complex areas, such as the Galactic plane. These tailored observational approaches are integral to fulfilling LSST's broader scientific mandate, offering critical data to complement the primary survey and enabling detailed analyses of both local and distant astronomical phenomena.

\subsection{Scientific Impact}

LSST is expected to make transformative contributions across multiple domains of astrophysics. In cosmology, it will provide precise measurements of dark energy parameters through studies of weak gravitational lensing, supernovae, baryon acoustic oscillations, and large-scale structure~\cite{2012arXiv1211.0310L}. In the study of the Milky Way, LSST will map the distribution and kinematics of stars, facilitating investigations into the Galaxy's formation history and structure~\cite{2019NatRP...1..450R}. The time-domain capabilities of LSST will enable the discovery and monitoring of transient and variable objects, such as supernovae, variable stars, and potentially hazardous near-Earth objects~\cite{2018Icar..303..181J}. Furthermore, LSST's extensive dataset will be invaluable for probing fundamental physics, such as testing general relativity on cosmological scales and searching for signatures of new physics beyond the Standard Model.
