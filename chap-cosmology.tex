In this chapter, we provide an overview of the fundamental concepts and equations that underpin the field of cosmology following textbooks \citet{2003moco.book.....D} and \citet{2008cosm.book.....W}.

\section{Einstein Field Equations}
The Einstein Field Equations are the fundamental equations of General Relativity, describing how matter and energy influence the curvature of spacetime. Introduced by Albert Einstein~\citep{1915SPAW.......844E}, these equations extend Newton's law of universal gravitation to a relativistic context, accounting for the effects of high velocities and strong gravitational fields.
The EFE establish a relationship between the geometry of spacetime and the distribution of matter within it. Mathematically, they are expressed as:
\begin{equation}
    G_{\mu\nu} + \Lambda g_{\mu\nu}= \frac{8\pi G}{c^4} T_{\mu\nu},
    \label{eq:einstein_field_equations}
\end{equation}
where \( G_{\mu\nu} \) denotes the Einstein tensor, defined by
\begin{equation}
    G_{\mu\nu} = R_{\mu\nu} - \frac{1}{2} R g_{\mu\nu},
    \label{eq:einstein_tensor}
\end{equation}
the $R_{\mu\nu}$ is the Ricci curvature tensor:
\begin{equation}
    R_{\mu\nu} = \partial_\lambda \Gamma^\lambda_{\mu\nu} - \partial_\nu \Gamma^\lambda_{\mu\lambda} + \Gamma^\lambda_{\lambda\sigma} \Gamma^\sigma_{\mu\nu} - \Gamma^\lambda_{\mu\sigma} \Gamma^\sigma_{\lambda\nu},
    \label{eq:ricci_curvature_tensor}
\end{equation}
the Ricci curvature scalar $R$ is given by:
\begin{equation}
    R = g^{\mu\nu} R_{\mu\nu},
    \label{eq:ricci_scalar}
\end{equation}
where \( g_{\mu\nu} \) is the metric tensor. The stress-energy tensor \( T_{\mu\nu} \) describes the distribution of matter and energy in spacetime.
Assuming a perfect fluid as the source of the gravitational field, the stress-energy tensor is given by
\begin{equation}
    T_{\mu\nu} = \left(\rho + p \right) u_{\mu} u_{\nu} + p g_{\mu\nu},
    \label{eq:stress_energy_tensor}
\end{equation}
where \( \rho \) is the energy density, \( p \) is the pressure, and \( u_{\mu} \) is the four-velocity of the fluid.
In a homogeneous and isotropic universe, \( u_{\mu} \) is given by
\begin{equation}
    u_{\mu} = (-c, 0, 0, 0),
    \label{eq:four_velocity}
\end{equation}
Therefore, each component of the stress-energy tensor can be expressed as
\begin{equation}
    T_{00} = \rho c^2, \quad T_{ij} = p g_{ij},
    \label{eq:stress_energy_components}
\end{equation}
Different speices of matter and energy contribute to the energy density \( \rho \) and pressure \( P \) in the universe. The equation of state parameter \( w \) is defined as the ratio of pressure \( P \) to energy density \( \rho \):
\begin{equation}
    w = \frac{p}{\rho}.
    \label{eq:equation_of_state}
\end{equation}
For perfect fluids, the equation of state parameter can be derived by considering the trace of the stress-energy tensor:
\begin{equation}
    0 = T = g^{\mu\nu} T_{\mu\nu} = (\rho + p)(-1) + 4p = -\rho + 3p,
    \label{eq:stress_energy_trace}
\end{equation}
For non-relativistic matter where $p = 0$, the equation of state parameter is \( w = 0 \). 
For dark energy, the equation of state parameter can be derived by comparing the effective stress-energy of the cosmological constant $T_{\mu\nu}^{(\Lambda)} = \rho_{\Lambda} g_{\mu\nu}$ to the stress-energy tensor of a perfect fluid:
\begin{align}
    \rho_{\Lambda} &= \frac{\Lambda c^4}{8\pi G} \label{eq:lambda_density} \\
    \left(\rho_{\Lambda} + p_{\Lambda} \right) u_{\mu} u_{\nu} + p_{\Lambda} g_{\mu\nu} &= T_{\mu\nu}^{(\Lambda)} = \rho_{\Lambda} g_{\mu\nu}  \\
    \rho_{\Lambda} + p_{\Lambda} &= 0 \quad \text{(valid for all $\mu$, $\nu$)} \nonumber \\
    p_{\Lambda} &= -\rho_{\Lambda} \label{eq:lambda_pressure}
\end{align}
Therefore, the equation of state parameter for dark energy is \( w = -1 \).

\section{FLRW Metric and the Friedmann Equations}
\label{sec:flrw_metric}
The Friedmann-Lemaître-Robertson-Walker (FLRW) metric describes a homogeneous and isotropic universe and is given by~\citet{1972gcpa.book.....W}:
\begin{equation}
    ds^2 = -c^2 dt^2 + a^2(t) \left[ d\chi^2 + f_K^2(\chi) \left( d\theta^2 + \sin^2\theta \, d\phi^2 \right) \right],
    \label{eq:flrw_metric}
\end{equation}
where \( a(t) \) is the scale factor, \( \chi \) is the comoving radial coordinate, and \( d\Omega^2 = d\theta^2 + \sin^2\theta \, d\phi^2 \) represents the metric on the unit two-sphere. The function \( f_K(\chi) \) encodes the spatial curvature of the universe and is defined as:
\begin{equation}
    f_K(\chi) = 
    \begin{cases}
        \dfrac{1}{\sqrt{-K}} \sinh\left(\sqrt{-K}\chi\right) & \text{for } K < 0, \\
        \chi & \text{for } K = 0, \\
        \dfrac{1}{\sqrt{K}} \sin\left(\sqrt{K}\chi\right) & \text{for } K > 0,
    \end{cases}
    \label{eq:fk_definition}
\end{equation}
where \( K \) is the spatial curvature constant, with \( K < 0 \) corresponding to an open universe, \( K = 0 \) to a flat universe, and \( K > 0 \) to a closed universe.

For the FLRW metric, the non-zero components of the Einstein tensor are:
\begin{align}
    G_{00} &= 3\left( \frac{\dot{a}}{a} \right)^2 + 3\frac{K c^2}{a^2}, \label{eq:G00_component} \\
    G_{ij} &= -\left( 2\frac{\ddot{a}}{a} + \left( \frac{\dot{a}}{a} \right)^2 + \frac{K c^2}{a^2} \right) a^2 g_{ij}, \label{eq:Gij_component}
\end{align}
where the dot denotes differentiation with respect to cosmic time \( t \).

Substituting the components of \( G_{\mu\nu} \) and \( T_{\mu\nu} \) into the Einstein field equations~\eqref{eq:einstein_field_equations}, we obtain the Friedmann equations~\citep{1922ZPhy...10..377F}:
\begin{itemize}
    \item \textbf{First Friedmann equation} (from the \( 00 \) component):
    \begin{equation}
        \left( \frac{\dot{a}}{a} \right)^2 = \frac{8\pi G}{3} \rho - \frac{K c^2}{a^2} + \frac{\Lambda c^2}{3},
        \label{eq:friedmann_first}
    \end{equation}
    \item \textbf{Second Friedmann equation} (from the \( ii \) components):
    \begin{equation}
        \frac{\ddot{a}}{a} = -\frac{4\pi G}{3} \left( \rho + \frac{3P}{c^2} \right) + \frac{\Lambda c^2}{3}.
        \label{eq:friedmann_second}
    \end{equation}
\end{itemize}
Introducing the Hubble parameter \( H \) and the critical density \( \rho_c \), we can simplify the Friedmann equations. The Hubble parameter is defined as:
\begin{equation}
    H = \frac{\dot{a}}{a},
    \label{eq:hubble_parameter}
\end{equation}
and the critical density is defined as:
\begin{equation}
    \rho_c = \frac{3 H^2}{8\pi G}.
    \label{eq:critical_density}
\end{equation}
Substituting \( H \) and \( \rho_c \) into the first Friedmann equation~\eqref{eq:friedmann_first}, we obtain:
\begin{equation}
    H^2 = H^2 \frac{\rho}{\rho_c} - \frac{K c^2}{a^2} + \frac{\Lambda c^2}{3}.
    \label{eq:friedmann_with_critical_density}
\end{equation}
Rearranging terms, we get:
\begin{equation}
    1 = \frac{\rho}{\rho_c} - \frac{K c^2}{a^2 H^2} + \frac{\Lambda c^2}{3 H^2}.
    \label{eq:friedmann_normalized}
\end{equation}
Defining the density parameters:
\begin{equation}
    \Omega_m = \frac{\rho_m}{\rho_c}, \quad \Omega_r = \frac{\rho_r}{\rho_c}, \quad \Omega_\Lambda = \frac{\rho_\Lambda}{\rho_c} = \frac{\Lambda c^2}{3 H^2}, \quad \Omega_K = -\frac{K c^2}{a^2 H^2},
    \label{eq:density_parameters}
\end{equation}
where \( \rho_m \) and \( \rho_r \) are the energy densities of matter and radiation, respectively, and \( \rho_\Lambda \) is the effective energy density associated with the cosmological constant, we can write the first Friedmann equation as:
\begin{equation}
    1 = \Omega_r + \Omega_m + \Omega_K + \Omega_\Lambda.
    \label{eq:friedmann_density_parameters}
\end{equation}
The evolution of the density parameters with the scale factor \( a \) can be derived from the conservation of energy-momentum and the equations of state. For matter-dominated and radiation-dominated universes, the energy densities scale as:
\begin{equation}
    \rho_m \propto a^{-3}, \quad \rho_r \propto a^{-4}.
    \label{eq:energy_density_scaling}
\end{equation}
Therefore, the corresponding density parameters vary with \( a \) as:
\begin{equation}
    \Omega_m(a) = \Omega_{m0} a^{-3} \left( \frac{H_0}{H(a)} \right)^2, \quad \Omega_r(a) = \Omega_{r0} a^{-4} \left( \frac{H_0}{H(a)} \right)^2,
    \label{eq:density_parameters_evolution}
\end{equation}
where the subscript \( 0 \) denotes present-day values, and \( H_0 \) is the current Hubble parameter.

Combining these expressions, the Friedmann equation~\eqref{eq:friedmann_density_parameters} can be rewritten in terms of the present-day density parameters:
\begin{equation}
    \left( \frac{H(a)}{H_0} \right)^2 = \Omega_{r0} a^{-4} + \Omega_{m0} a^{-3} + \Omega_{K0} a^{-2} + \Omega_{\Lambda0},
    \label{eq:friedmann_rewritten}
\end{equation}
which describes the evolution of the Hubble parameter with scale factor \( a \) in terms of the contributions from radiation, matter, curvature, and the cosmological constant.

\section{The linear evolution of density fluctuations}
Linear perturbation theory provides a fundamental framework for understanding the evolution of small deviations from the homogeneous and isotropic background of the Universe. Starting from the continuity and Euler equations, which govern the conservation of mass and momentum, respectively:
\begin{eqnarray}
    \label{eq:continuity_equation}
    \dfrac{\partial \rho}{\partial t} + \nabla \cdot (\rho \boldsymbol{v}) &=& 0, \\[2ex]
    \label{eq:euler_equation}
    \dfrac{\partial \boldsymbol{v}}{\partial t} + (\boldsymbol{v} \cdot \nabla) \boldsymbol{v} &=& -\dfrac{\nabla P}{\rho} - \nabla \Phi,
\end{eqnarray}
where \( \rho \) is the density, \( \boldsymbol{v} \) is the peculiar velocity field, \( P \) is the pressure, and \( \Phi \) is the gravitational potential.

To analyze perturbations in an expanding universe, we move to comoving coordinates and express the density as a perturbation around the mean density, \( \rho = \bar{\rho}(1 + \delta) \), where \( \delta \) is the density contrast. The continuity and Euler equations then become:
\begin{eqnarray}
    \label{eq:continuity_equation_rewritten}
    \dfrac{\partial \delta}{\partial t} + \dfrac{1}{a} \nabla \cdot \left( (1 + \delta) \boldsymbol{v} \right) &=& 0, \\[2ex]
    \label{eq:euler_equation_rewritten}
    \dfrac{\partial \boldsymbol{v}}{\partial t} + H \boldsymbol{v} + \dfrac{1}{a} (\boldsymbol{v} \cdot \nabla) \boldsymbol{v} &=& -\dfrac{\nabla \delta P}{a \bar{\rho} (1 + \delta)} - \dfrac{1}{a} \nabla \Phi,
\end{eqnarray}
where \( a(t) \) is the scale factor, and \( H = \dot{a}/a \) is the Hubble parameter.

To derive the equation of motion for the density contrast, we linearize the above equations under the assumption that \( \delta \ll 1 \) and \( \boldsymbol{v} \) is small. Neglecting higher-order terms in \( \delta \) and \( \boldsymbol{v} \), we obtain the linearized equations:
\begin{eqnarray}
    \label{eq:linear_continuity}
    \dfrac{\partial \delta}{\partial t} + \dfrac{1}{a} \nabla \cdot \boldsymbol{v} &=& 0, \\[2ex]
    \label{eq:linear_euler}
    \dfrac{\partial \boldsymbol{v}}{\partial t} + H \boldsymbol{v} &=& -\dfrac{\nabla \delta P}{a \bar{\rho}} - \dfrac{1}{a} \nabla \Phi.
\end{eqnarray}
The gravitational potential \( \Phi \) is related to the density contrast via Poisson's equation in comoving coordinates:
\begin{equation}
    \label{eq:poisson_equation}
    \nabla^2 \Phi = 4\pi G \bar{\rho} a^2 \delta,
\end{equation}
where \( G \) is the gravitational constant.
Assuming adiabatic perturbations, the pressure perturbation is related to the density perturbation by \( \delta P = c_s^2 \delta \rho = c_s^2 \bar{\rho} \delta \), where \( c_s \) is the sound speed.

Taking the time derivative of the linearized continuity equation (\ref{eq:linear_continuity}) and substituting the divergence of \( \boldsymbol{v} \) from the linearized Euler equation (\ref{eq:linear_euler}), we obtain:
\begin{equation}
    \dfrac{\partial^2 \delta}{\partial t^2} + 2H \dfrac{\partial \delta}{\partial t} - \dfrac{c_s^2}{a^2} \nabla^2 \delta = 4\pi G \bar{\rho} \delta.
\end{equation}
Transforming to Fourier space, where \( \nabla^2 \delta \rightarrow -k^2 \tilde{\delta}(k, t) \), the equation becomes:
\begin{equation}
    \label{eq:delta_fourier}
    \ddot{\tilde{\delta}}(k, t) + 2H \dot{\tilde{\delta}}(k, t) + \left( \dfrac{c_s^2 k^2}{a^2} - 4\pi G \bar{\rho} \right) \tilde{\delta}(k, t) = 0,
\end{equation}
where \( \tilde{\delta}(k, t) \) is the Fourier transform of the density contrast.

Defining the effective frequency squared \( \omega^2(k, t) = 4\pi G \bar{\rho} - \dfrac{c_s^2 k^2}{a^2} \), the equation simplifies to:
\begin{equation}
    \ddot{\tilde{\delta}}(k, t) + 2H \dot{\tilde{\delta}}(k, t) - \omega^2(k, t) \tilde{\delta}(k, t) = 0.
\end{equation}
The solutions to this differential equation depend on the sign of \( \omega^2(k, t) \):
\begin{itemize}
    \item \textbf{Gravity-Dominated Regime (\( \omega^2(k, t) > 0 \))}: For large-scale perturbations where gravity overcomes pressure forces (i.e., small \( k \)), the solutions are exponential:
    \begin{equation}
        \label{eq:delta_growing_mode}
        \tilde{\delta}(k, t) = C_1 e^{\lambda t} + C_2 e^{-\lambda t},
    \end{equation}
    where \( \lambda = \sqrt{\omega^2(k, t)} \). The growing mode (\( e^{\lambda t} \)) leads to the amplification of perturbations and structure formation, while the decaying mode (\( e^{-\lambda t} \)) becomes negligible over time.
    \item \textbf{Pressure-Dominated Regime (\( \omega^2(k, t) < 0 \))}: For small-scale perturbations where pressure resists gravitational collapse (i.e., large \( k \)), the solutions are oscillatory:
    \begin{equation}
        \label{eq:delta_oscillatory_mode}
        \tilde{\delta}(k, t) = e^{-Ht} \left( C_1 \cos{\left( |\omega(k, t)| t \right)} + C_2 \sin{\left( |\omega(k, t)| t \right)} \right).
    \end{equation}
    The perturbations oscillate with frequency \( |\omega(k, t)| \) and are damped by the cosmic expansion, preventing collapse on small scales.
\end{itemize}
These results illustrate the Jeans instability criterion, which states that perturbations grow only if their wavelength exceeds the Jeans length \( \lambda_J = c_s \sqrt{\dfrac{\pi}{G \bar{\rho}}} \) \citep{1902RSPTA.199....1J}.

\section{Cosmological Distances}
For light-like (null) geodesics, the spacetime interval \( ds^2 \) is zero. Thus, the radial coordinate distance for a photon traveling from a source to the observer is obtained from the null condition:
\begin{equation}
    ds^2 = 0 \quad \Rightarrow \quad d\chi = \frac{c \, dt}{a(t)}.
    \label{eq:null_geodesic_condition}
\end{equation}
Integrating this expression, we obtain the comoving radial distance \( \chi(z) \) as a function of redshift \( z \):
\begin{equation}
    \chi(z) = \int_{t(z)}^{t_0} \frac{c \, dt'}{a(t')} = \int_{0}^{z} \frac{c \, dz'}{H(z')},
    \label{eq:comoving_radial_distance}
\end{equation}
where \( t_0 \) is the present time. The redshift \( z \) is related to the scale factor by \( 1 + z = \tfrac{a_0}{a(t)} \), with \( a_0 \equiv a(t_0) = 1 \) for the present universe.

In the late universe, where radiation is negligible compared to matter and dark energy, the Hubble parameter \( H(z) \) is given by the Friedmann Eq.~\eqref{eq:friedmann_rewritten}:
\begin{equation}
    H(z) = H_0 \sqrt{\Omega_{m0}(1+z)^3 + \Omega_{K0}(1+z)^2 + \Omega_{\Lambda0}},
    \label{eq:hubble_parameter_late_universe}
\end{equation}
where \( H_0 \) is the present-day Hubble constant, and \( \Omega_{m0} \), \( \Omega_{K0} \), and \( \Omega_{\Lambda0} \) are the present-day density parameters for matter, curvature, and dark energy, respectively.

\subsection{Luminosity Distance}
The luminosity distance \( d_L(z) \) is a key concept in observational cosmology, relating the intrinsic luminosity \( L \) of an astronomical object to the observed flux \( F \) via the inverse-square law~\citep{1992ARA&A..30..499C}:
\begin{equation}
    F = \frac{L}{4\pi d_L^2}.
    \label{eq:luminosity_flux}
\end{equation}
In an expanding universe, the luminosity distance accounts for the effects of redshift on both the energy of photons and the rate at which they are received. It is defined as~\citep{1999astro.ph..5116H}:
\begin{equation}
    d_L(z) = (1 + z) \, f_K(\chi(z)).
    \label{eq:luminosity_distance}
\end{equation}
The luminosity distance is crucial for determining cosmological parameters using standard candles, such as Type Ia supernovae, whose intrinsic luminosities are known~\cite{1998AJ....116.1009R}. By measuring the observed flux \( F \) and applying Eq.~\eqref{eq:luminosity_flux}, we can infer \( d_L(z) \) and constrain cosmological models.

\subsection{Angular Diameter Distance}
The angular diameter distance \( d_A(z) \) relates the physical size \( D \) of an object to its observed angular size:
\begin{equation}
    \theta = \frac{D}{d_A}.
    \label{eq:angular_diameter}
\end{equation}
In an expanding universe, the angular diameter distance is given by~\citep{1999astro.ph..5116H}:
\begin{equation}
    d_A(z) = \frac{f_K(\chi(z))}{1 + z}.
    \label{eq:angular_diameter_distance}
\end{equation}
The angular diameter distance is essential for studying standard rulers, such as the scale of baryon acoustic oscillations (BAO) in the cosmic microwave background and large-scale structure~\citep{2005ApJ...633..560E}. By measuring the angular size \( \theta \) of these features and knowing their physical size \( D \), we can determine \( d_A(z) \) and thus constrain cosmological parameters.