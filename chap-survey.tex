\section{Introduction to Astronomical Surveys}
Astronomical surveys are extensive observational projects designed to map large regions of the sky with high depth and precision, producing critical datasets for fundamental questions in astrophysics and cosmology. They aim to test the standard cosmological model ($\Lambda$CDM) by providing precise measurements that can confirm or challenge it, addressing issues like the \emph{Hubble tension}—a discrepancy in expansion rate measurements from early~\citep{2016A&A...594A..13P} and late~\citep{2019ApJ...876...85R} observations—and inconsistencies in parameters like $S_8$. Surveys also study the formation and evolution of cosmic structures by mapping millions of galaxies and dark matter distributions using techniques like \emph{weak gravitational lensing} and \emph{galaxy-galaxy lensing}\citep{2013MNRAS.432.1544M, 2022PhRvD.105b3520A}.

These surveys employ different methodologies: \emph{imaging surveys} capture wide-field images across multiple wavelengths to map cosmic structures and analyze galaxy populations (e.g., SDSS~\citep{2019BAAS...51g.274K}, DES~\citep{2018ApJS..239...18A}, LSST~\citep{2019ApJ...873..111I}), while \emph{spectroscopic surveys} collect spectral data revealing redshifts, compositions, and kinematics essential for studying galaxy dynamics and the universe's expansion (e.g., BOSS~\citep{2013AJ....145...10D}, DESI~\citep{2016arXiv161100036D}, KiDS with spectroscopic extensions~\citep{2013Msngr.154...44D}). They can be \emph{ground-based}, utilizing Earth-based telescopes but limited by atmospheric effects (e.g., HSC~\citep{2018PASJ...70S...4A}, DES, KiDS), or \emph{space-based}, operating above Earth's atmosphere for higher clarity and sensitivity, especially in inaccessible wavelengths (e.g., HST~\citep{2001ApJ...553...47F}, the upcoming \emph{Nancy Grace Roman Space Telescope}\citep{2015arXiv150303757S}, and the \emph{Gaia} mission\citep{2016A&A...595A...2G}).

Surveys are also classified into \emph{Stage-III} and \emph{Stage-IV} based on technological sophistication and scale~\citep{2006astro.ph..9591A}. \emph{Stage-III} surveys (e.g., DES, KiDS, HSC) represent the current generation aiming to refine cosmological parameters and deepen understanding of dark energy and dark matter. \emph{Stage-IV} surveys (e.g., Rubin Observatory~\citep{2019ApJ...873..111I}, DESI, the upcoming \emph{Roman Space Telescope}) are the next generation characterized by unprecedented scale and precision, aiming for high-precision cosmological measurements and deeper exploration of dark energy and dark matter.

\section{Overview of Major Galaxy Surveys}
Several significant galaxy surveys have been designed to measure weak lensing signals with high precision. In this section, we provide a comprehensive overview of four pivotal surveys focusing on their observational capabilities.
\begin{table}[h]
    \centering
    \caption{Comparison of Key Galaxy Surveys for Weak Lensing}
    \label{tab:survey_comparison}
    \begin{tabular}{lccc}
        \toprule
        \textbf{Survey} & \textbf{Area (deg$^2$)} & \textbf{Approx. Galaxy Density (arcmin$^{-2}$)} & \textbf{Median Redshift} \\
        \midrule
        DES & $\sim$5,000 &  7 & 0.4 \\
        HSC Wide & $\sim$1,400 &  15 & 0.7 \\
        LSST & $\sim$18,000 & 30 & 1.0 \\
        Roman Telescope & $\sim$2,000 & $50$ & 1.5 \\
        \bottomrule
    \end{tabular}
\end{table}

The Dark Energy Survey (DES; \citealt{2005astro.ph.10346T, 2018ApJS..239...18A, 2021ApJS..255...20A}) utilized the 570-megapixel Dark Energy Camera (DECam; \citealt{2015AJ....150..150F}) mounted on the 4-m Blanco Telescope at the CTIO in Chile. Over the course of its operation, DES observed more than 300 million galaxies across approximately 5,000~deg$^2$ of the southern sky in five optical bands ($g$, $r$, $i$, $z$, $Y$). It achieved an effective galaxy density of about $\sim 6$~arcmin$^{-2}$ and provided photometric redshift estimates up to $z \sim 1.2$. The data collected by DES has made significant contributions to cosmology and astrophysics, including precise measurements of cosmic shear~\citep{2022PhRvD.105b3514A} and galaxy clustering~\citep{2022PhRvD.105b3520A}. 

The Hyper Suprime-Cam Subaru Strategic Program (HSC-SSP; \citealt{2018PASJ...70S...4A}) comprises three layers: Wide, Deep, and UltraDeep, conducted with the 8.2-m Subaru Telescope equipped with the 870-megapixel Subaru Hyper Suprime-Cam (HSC; \citealt{2018PASJ...70S...1M}). The Wide layer covers approximately 1,400~deg$^2$ and reaches a depth of $i \sim 26$, yielding galaxy densities of around $\sim 15$~arcmin$^{-2}$. Photometric redshifts extend up to $z \approx 2$. The superior imaging quality of HSC enhances the accuracy of weak lensing measurements and contributes to tighter cosmological constraints \citep{2019PASJ...71...43H}.

The Legacy Survey of Space and Time (LSST; \citealt{2009arXiv0912.0201L, 2019ApJ...873..111I}) is conducted at the Vera C. Rubin Observatory. Over a 10-year period, LSST will survey approximately 18,000~deg$^2$ of the sky, reaching a depth of $r \sim 27.5$. It is expected to detect around 20~billion galaxies, corresponding to galaxy densities exceeding $\sim 30$~arcmin$^{-2}$, with redshift measurements up to $z \approx 3$. LSST's vast dataset will substantially improve the statistical precision of weak lensing analyses and further refine cosmological models \citep{2012arXiv1211.0310L}.

The \textit{Nancy Grace Roman Space Telescope} (\emph{Roman}; \citealt{2015arXiv150303757S}) will conduct wide-field near-infrared imaging and spectroscopy from space scheduled for launch in the mid-2020s. Covering approximately 2,000~deg$^2$, it will reach a magnitude limit of $H \approx 26.7$. The expected galaxy densities exceed $\sim 50$~arcmin$^{-2}$, facilitated by its space-based observations. The mission aims to provide spectroscopic redshifts higher than $z \approx 3$, significantly enhancing the precision of weak lensing measurements.