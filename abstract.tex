We investigate the influence of super-sample covariance on the covariance and correlation matrices of the power spectrum and higher-order statistics using two distinct simulation approaches: BIGBOX and TILED. Leveraging 11 realizations of state-of-the-art BIGBOX simulations from the HalfDome project, each containing $6144^3$ particles within a 3750 Mpc/h volume, and 20 TILED realizations composed of smaller $1024^3$-particle boxes within 625 Mpc/h, we analyze the convergence field under the Born approximation. Uniformly covering the full sky with $10^\circ \times 10^\circ$ patches extracted via a Fibonacci grid, we employ a comprehensive suite of higher-order statistics, including bispectrum, probability distribution function, peak and minima counts, and Minkowski functionals, and the angular power spectrum to probe the non-Gaussian features of the convergence maps.

Our comparative analysis reveals that mean statistical measures between BIGBOX and TILED simulations agree within 1\% across all statistics, redshifts, and multipoles, validating the simulations' reliability in capturing the underlying cosmological model. However, variance ratios exhibit significant differences of 10-30\%, particularly at high multipoles and redshifts, attributable to super-sample covariance arising from large-scale modes present in BIGBOX but absent in TILED. Shape noise impacts statistical measures, especially the angular power spectrum and peak/minima counts, yet super-sample covariance remains the dominant effect. Additionally, Gaussian smoothing affects small-scale features without substantially variating the covariance and correlation structures, while box replication artifacts introduce systematic biases by underestimating mean angular power spectra and distorting higher-order statistics.

These findings underscore the critical role of super-sample covariance in high-precision cosmological analyses, particularly for upcoming weak lensing surveys such as LSST, Euclid, and Roman. Accurate variance modeling necessitates the inclusion of large-scale modes in simulations, while maintaining robustness against shape noise and smoothing operations. Addressing systematic challenges like box replication artifacts is essential to prevent biases in the interpretation of higher-order statistics. Future work will focus on mitigating replication artifacts, incorporating diverse cosmological models and baryonic effects, and integrating these insights into survey design and data analysis pipelines to enhance the precision of cosmological parameter inference. Our study advances simulation techniques and observational strategies, reinforcing weak lensing as a pivotal tool for probing the Universe's fundamental properties with unprecedented accuracy.